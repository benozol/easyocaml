
\section{Errors and Error Reporting Adaptibility}
\label{hd003}
EasyOCaml is essentially build for teaching programming .  Therefore, 
special attention is paid to the way errors are reported.

Firstly, errors should provide a \emph{right} amount of details, too few 
information is of course insufficient, but also too much information can 
be confusing. So, for example in type constructor clashes exactly those 
locations are reported, which are essential to the error.  Delivering 
more information on the reasons of type errors is exactly what 
EasyOCaml's type checker is made for.

Secondly, error reporting should be adaptable: For a beginner, reading 
errors in a foreign language can distract or even prevent him from 
fixing an error. Furthermore, the error output should be adaptable in 
its overall structure to serve as the input for different kinds of 
presentation, e.g. plain text on command line or HTML for visually 
display it in a web browser.

The last section has explained the improvements of EasyOCaml to type 
error messages. The following will describe improvements to the parser.


\subsection{New Errors for Camlp4}
\label{hd003001}
As mentioned, EasyOCaml parses its input with a Camlp4 parser.  
Unfortunately, Camlp4's error messages are hard coded in the parser's 
code in English and never represented in data.  The reason is that 
Camlp4 is a OCaml stream parser in its core, and this requires parsing 
errors to be reported as \texttt{Stream.Error~of~string} exceptions.

Nevertheless, we supplied Camlp4 with a new error reporting system, up 
to now just to make error reporting adaptable, but it should be possible 
now to augment the information of parse errors by the parser's current 
state. Now, a parsing error \texttt{ParseError.t} is one of the 
following:

\begin{description}
    \item[\texttt{Expected~(entry,~opt\_before,~context)}] is raised 
        if when the parser stucks while parsing a phrase: \texttt{entry} 
        describe the categories of the possible, expected subphrases, 
        \texttt{opt\_before} might describe the category of the entry 
        just parsed and \texttt{context} denotes the category of the 
        phrase currently parsed.
    \item[\texttt{Illegal\_begin~sym}] is raised when the parser is 
        not able to parse the top categories described by \texttt{sym}.
    \item[\texttt{Failed}] is raised only in 
        \texttt{Camlp4.Struct.Grammar.Fold}.
    \item[\texttt{Specific\_error~err}] Beside the generic parsing 
        errors just mentioned, it is possible to extend the parsing 
        errors per language by ``artificial'' errors which are specific 
        to a language, e.g. \texttt{Currified~constructor} in OCaml, 
        which is not represented in the grammar but checked in code. 
        (further errors for EasyOCaml are specified in 
        \texttt{Camlp4.Sig.OCamlSpecificError}.)
\end{description}

Now, how are these errors represented in the string information of the 
\texttt{Stream.Error}?  Not without a hack, which is luckily hidden 
behind the interface of Camlp4:  Internally, Camlp4 throws 
\texttt{Stream.Error} exceptions but the string has the following 
format: ``\texttt{$<$msg$>$\textbackslash{}000$<$mrsh$>$}'' where 
\texttt{$<$msg$>$} is the usual Camlp4 error message and 
\texttt{$<$mrsh$>$} is a marshalled parsing error as just described. The 
string of a \texttt{Stream.Error} is again decomposed in the interface 
function for parsing (namely 
\texttt{Camlp4.Struct.Grammar.Entry.action\_parse}), and reported as a 
\texttt{ParseError.t} to the user.


\subsection{Adaptibility}
\label{hd003002}
For internationalization of error messages and different structure of 
error messages for different display settings, EasyOCaml provides 
adaptability of error messages by a plugin system.  Error reporting 
plugins should use \texttt{EzyError}`s internationalized functions to 
output the error's description (\texttt{EzyErrors.print*\_desc}) to keep 
them uniform but can print it any structure:  Currently, a plain text 
format is default and a HTML printer which highlights the locations of 
an error in source code and a XML/Sexp printer for usage in an IDE are 
delivered with EasyOCaml.

The user can register an error printer via the command line flag 
\texttt{-easyerrorprinter}.  The module is dynamically linked and 
registers itself with \texttt{EzyErrors.register} where appropriate 
functions are overwritten.

The following section describes the tools, EasyOCaml provides 
specifically for teaching programming.
