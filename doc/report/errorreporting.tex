
\section{Errors and Error Reporting Adaptability}
\label{sec:errors}

\easyocaml\ is essentially build for teaching programming.
As such, special attention is paid to the way errors are reported to achieve
the following goals:

First, errors should provide a right amount of details: Needless to say that too
few information are insufficient.
But too much information can be just as confusing.
So, for example in type constructor clashes exactly those locations
of the program should be reported which are essential to the error.
Delivering the right amount of information on the underlying reasons of a type
error is exactly what \easyocaml's type checker is made for.

Second, error reporting should be adaptable: Reading errors in a foreign
language can distract a beginner or even prevent him from understanding it.
So internationalization of error messages is necessary.
Furthermore, the error output should be adaptable in its overall structure to
serve for different kinds of presentation, e.g.  on command line, or in a web
browser.

The last section has explained the improvements of \easyocaml\ to the type
error messages. This section will first describe the structure  of the errors
handled by
\easyocaml\ as well as the changes to the parser to build a foundation to
include more information with the parsing errors, and then the error adaption
possibilities.

\subsection{The Structure of Errors in EasyOCaml}
\label{sec:easyerrors}

While parsing and type checking a program, \easyocaml\ can detect different errors.
Those errors can be separated into three classes differing by the stage of the
compilation process where they occur and in the impact on the subsequent
compilation process.
This section describes the error classes and gives some examples for the
participating errors.  The file \texttt{ezyErrors.mli} provides a complete and
well documented list of the errors.

The compiler attempts to report errors as late as possible to collect as many
errors as possible. \emph{Common errors} are reported at the very end, i.e.\
after constraint unification. So type errors (type constructor clashes, clashes
of the arities of tuples and circular types) are quite natural of this kind.
Furthermore, unbound variables are admittedly detected during constraint
generation but provided with free type variables, such that constraint
unification is possible without introducing any artificial errors.
Other light errors like the attempt to change the value of immutable record
fields are reported after constraint unification, too, as well as invalid type
annotations.

\emph{Heavy errors} are collected during and reported directly after constraint
generation, because they prohibit even a transitionally type assignment 
to the corresponding term. They include among others:
Incoherent record constructions (e.g.\ fields from different records or several
bindings of a single field), several bindings of the same variable in a pattern,
the usage of unknown variant constructors or the wrong usage of type
constructors (unknown or wrong number of type arguments).

Syntactical errors (parsing errors) and accessing inexistent modules are
\emph{fatal errors} and stop the compiler immediately because they prevent
any reasonable continuation of the compilation process.

The next section will describe our changes to \camlpf\ for more detailed
parsing errors.

\subsection{New Errors for Camlp4}

We have chosen \camlpf\ as the parser generator for \easyocaml\ because it is
the only system known to the authors that facilitates manipulation of the
grammar at runtime: The whole grammar is defined as an \ocaml\ program which
dynamically generates an \ocaml\ stream parser \citep{ocamlstreamparser}. Thus,
it is possible to modify the grammar by deleting certain rules at runtime. We
use this in the first place to prune a full-fledged \ocaml\ parser which is
shipped with \camlpf\ to \camlm\ and then to implement the subset of the
language specified by the language level.

But the foundation of \camlpf\ as an \ocaml\ stream parser yields to another
problem: Stream parsers allow only a single exception type for passing parsing
errors (internally and in the user interface). This exception can contain
just a string for information on the detected error (\texttt{exception
Stream.Error of string}). \camlpf's actual phrasing of the errors is hard-coded
deep in the parser code in the English language which prevents the
internationalization and format adaption necessary for \easyocaml.


Nevertheless, we supplied \camlpf\ with a new error reporting system, up 
to now just to make error reporting adaptable. But now it should be possible to
augment the information of parsing errors with more information on the state of
the parser.

Errors are represented in a variant type distinguishing the following forms of
parsing errors.
\begin{description}
    \item[\texttt{Expected~(entry,~opt\_before,~context)}] describes the most
      common error when the parser recognizes a certain syntactic category in
      the program which does not match the category given by the grammar.
      \texttt{entry} describes the expected category, \texttt{opt\_before}
      optionally describes the category of the entry just parsed and
      \texttt{context} denotes the category of the phrase which contains the
      questionable entry.
    \item[\texttt{Illegal\_begin~sym}] is raised when the parser is 
      not even able to parse the program's toplevel category denoted by
      \texttt{sym}.
    \item[\texttt{Failed}] is raised only in 
      \texttt{Camlp4.Struct.Grammar.Fold} and included in the error type just
      for consistency.
    \item[\texttt{Specific\_error~err}] is raised for language-specific,
      ``artificial'' errors.  Every grammar is parameterized on the type of
      \texttt{err} in our fork of \camlpf. This type covers conflicts with
      the language which can not be expressed by the \camlpf\ grammar but are
      checked in plain \ocaml\ code while parsing. 
      This includes three errors for \ocaml:
      Currified constructor, errors forcing an expression to be an identifier
      and bad directives on the REPL (the corresponding error is given in
      \texttt{Camlp4.Sig.OCamlSpecificError}).
\end{description}

But how are these errors represented in the string information of the stream
error? 
Not without a hack which is luckily hidden behind the interface of \camlpf:
Internally, parsing exceptions contain a string with format
``\texttt{<msg>\textbackslash 000<mrsh>}'' where \texttt{<msg>} is
the usual \camlpf\ error message and \texttt{<mrsh>} is the marshaled version
of the parsing error as just described.
This string is decomposed again in \camlpf's interface function for
parsing\ttfootnote[function ]{Camlp4.Struct.Grammar.Entry.action\_parse}, and
reported as a parsing error to the user.

So \camlpf's parsing errors are now represented in structured data to
apply \easyocaml's error reporting adaptabilities to them and to further extend
them with more information on the parser's state.

\subsection{Adaptability}

For internationalization of error messages and adjusting the structure of the
error messages to different display settings, \easyocaml\ provides two
possibilities to adapt the error reporting.

First, the language of the error messages is chosen via an environment variable.
The functions for phrasing the errors in different languages are part of
\easyocaml's error system. This has two reasons. On the one hand, \ocaml's
format adaption does not capture different orders of the arguments. This has
forced us to provide this functionality in \ocaml\ program code.
On the other hand, keeping the actual phrasing in the mainstream code suggests
the error format plugins to use common phrases for the messages.
This simplifies comparisons of errors under different display settings.
See section~\ref{sec:envvars} for more details on choosing the language.

Second, the format of the error messages can be adapted by so-called ``error
reporting plugins''.
A plugin consists of \ocaml\ code which defines arbitrary error formatting
functions and registers them in \easyocaml's error system.
As mentioned, those functions should use the provided functions for phrasing the
errors to enforce conformity. But they can print them in any structure.
Currently, a plain text formatter is the default for the usage on a command
line.
A HTML printer which highlights the locations of an error in the source code
with funny colors is delivered with \easyocaml\ for clear display with a
web browser, and a XML/Sexp printer for future usage in an IDE.
See section~\ref{sec:errorfmt} for more details on providing custom error
reporting plugins.

The following section describes the tools EasyOCaml provides specifically for
teaching programming.
