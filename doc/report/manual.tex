
\section{The User Interface (Manual)}
\label{sec:manual}

This section describe the extensions of the \texttt{ocaml} and \texttt{ocamlc} programs.

\subsection{Command Line Parameters}
\label{sec:commandlineflags}

\begin{description}
  \item[\texttt{-easy}] This flag enables \easyocaml\ and is obligatory for
    the usage of all other command line flags desribed here.
    It enables an alternative type checking algorithm wich gives more
    information on the type errors.
  \item[\texttt{-easyerrorprinter <printer>}] Loads an error formatting plugin
    into the compiler. This must be an \ocaml\ object that calls functions as
    given in section \ref{sec:errorfmt}.
  \item[\texttt{-easylevel <level>}] Enables a language level named
    \texttt{<teachpack>} which is installed in a configuration directory (see
    section \ref{sec:directory}).
  \item[\texttt{-easyteachpack <teachpack>}] Enables a teach pack named
    \texttt{<teachpack>} which is installed in a configuration directory (see
    section \ref{sec:directory}).
\end{description}

\subsection{Environment Variables}

\begin{description}
  \item[\texttt{EASYOCAML\_ENUM\_TIMEOUT}] \label{man:timeout}
    The real value controls the maximal amount of
    time \easyocaml\ may use to enumerate type errors (note, that the underlying algorithm
    has exponential time consumptions).
  \item[\texttt{EASYOCAML\_ONLY\_TYPECHECK}] Exit the compiler after type checking without any code generation.
  \item[\texttt{LANG}, \texttt{LANGUAGE}]  Controls internationalization of error messages. Possible
    values: ``en'', ``fr'', ``de''.
  \item[\texttt{EASYOCAML\_LOGLEVEL}] Controls debugging details. Possible values are ``error'', ``warn'', ``info'', ``debug'', and ``trace''.
  \item[\texttt{EASYOCAML\_GLOBAL\_DIR}] Specifies the global configuration directory as described below.
  \item[\texttt{EASYOCAML\_USER\_DIR}] Specifies the user configuration directory as described below.
\end{description}

\subsection{Defining Languge Levels}
TBD

\subsection{Defining Error Formatter}
\label{sec:errorfmt}

Error formatter for \easyocaml\ are given in \ocaml\ code. They must call
\texttt{EzyErrors.register} to register functions to print common, heavy and
fatal errors. After compiling the code with access to \easyocaml's code, the
resulting object file can be loaden into the compiler at runtime.

It is recommended to use the functions \texttt{EzyErrors.print\_errors},
\texttt{EzyErrors.print\_heavies} and \texttt{EzyErrors.print\_fatal} to
actually phrase the error and only define the layout with the plugin. This is
favorable, because those functions are already internationalized and provide a
common phrasing between the error printing plugins.

\subsection{The EasyOCaml Directory}
\label{sec:directory}

\easyocaml\ searches for language levels and teachpacks in a designated
configuration directory.

There is a global and a user configuration directory. First, EasyOCaml 
searches the user then the global configuration directory.  Here's how 
the global configuration directory is determined (in descending 
preference):

\begin{enumerate}
    \item Environment variable \texttt{EASYOCAML\_GLOBAL\_DIR}
    \item Compile-time option
\end{enumerate}

Here's how the user configuration directory is determined (in descending 
preference):

\begin{enumerate}
    \item Environment variable \texttt{EASYOCAML\_USER\_DIR}
    \item \texttt{\$HOME/.easyocaml}
\end{enumerate}

\easyocaml's configuration directory must have the following structure:

\begin{verbatim}
 lang-levels/level-1
             level-2
             ...
 teachpacks/tp-1
            tp-2
            ...
\end{verbatim}

Each language level and teach pack contains a module \texttt{LANG\_META} 
which is loaded into EasyOCaml.

