
\section{The User Interface (Manual)}
\label{sec:manual}

This section describe the extensions of the \texttt{ocaml} and \texttt{ocamlc} programs.

\subsection{Command Line Parameters}
\label{sec:commandlineflags}

\begin{description}
  \item[\texttt{-easy}] This flag enables \easyocaml\ and is obligatory for
    the usage of all other command line flags desribed here.
    It enables an alternative type checking algorithm wich gives more
    information on the type errors.
  \item[\texttt{-easyerrorprinter <printer>}] Loads an error formatting plugin
    into the compiler. This must be an \ocaml\ object that calls functions as
    given in section \ref{sec:errorfmt}.
  \item[\texttt{-easylevel <level>}] Enables a language level named
    \texttt{<teachpack>} which is installed in a configuration directory (see
    section \ref{sec:directory}).
  \item[\texttt{-easyteachpack <teachpack>}] Enables a teach pack named
    \texttt{<teachpack>} which is installed in a configuration directory (see
    section \ref{sec:directory}).
\end{description}

\subsection{Environment Variables}
\label{sec:envvars}

\begin{description}
  \item[\texttt{EASYOCAML\_ENUM\_TIMEOUT}] \label{man:timeout}
    The real value controls the maximal amount of
    time \easyocaml\ may use to enumerate type errors (note, that the underlying algorithm
    has exponential time consumptions).
  \item[\texttt{EASYOCAML\_ONLY\_TYPECHECK}] Exit the compiler after type checking without any code generation.
  \item[\texttt{LANG}, \texttt{LANGUAGE}]  Controls internationalization of error messages. Possible
    values: ``en'', ``fr'', ``de''.
  \item[\texttt{EASYOCAML\_LOGLEVEL}] Controls debugging details. Possible values are ``error'', ``warn'', ``info'', ``debug'', and ``trace''.
  \item[\texttt{EASYOCAML\_GLOBAL\_DIR}] Specifies the global configuration directory as described below.
  \item[\texttt{EASYOCAML\_USER\_DIR}] Specifies the user configuration directory as described below.
\end{description}

\subsection{Defining Languge Levels and Teach Packs}

A\new\ language level must specify the syntax and the modules available to the user.
This is done in \easyocaml\ in the following way. Each language level consists
of a directory in the configuration directory as described in
section~\ref{sec:directory}. The name of the directory defines
the name of the language level. The directory must contain a file name
\texttt{LANG\_META.ml} which actually defines the language level. This is done
by calling the function \texttt{EzyLangLevel.configure}. This function expects
the specification of the available syntax and modules as arguments.

Syntactic features are given by record values for every syntactic category.
Functions to generate minimal and maximal specifications are available for
convenience.

The available modules are specified in the call to
\texttt{EzyLangLevel.configure} just by two lists. The first one is a list of
\ocaml\ module names paired with a boolean value designating if they are opened
on start-up (like \texttt{Pervasives} is by default). The second constists of
the file names which contain those modules.

Teach packs are defined in an appropriate directory as described in
section~\ref{sec:directory} and must contain a file called
\texttt{TEACHPACK\_META.ml}. The actual teach pack in defined by a call to
\texttt{EzyTeachpack.configure} which expects the modules descriptions in two
lists like just described.
 
Note that this fashion to define language levels and teach packs might change in
future versions of \easyocaml.  A plain text file is sufficient because \ocaml\
code is necessary neither to specify the syntax nor to specify the available
modules.

\subsection{Defining Error Formatter}
\label{sec:errorfmt}

Error formatter for \easyocaml\ are given in compiled \ocaml\ code. They must
call \texttt{EzyErrors.register} to register functions to print common, heavy
and fatal errors. After compiling the code with access to \easyocaml's code,
the resulting object file can be loaden into the compiler at runtime with help
of the \texttt{-easyerrorprinter} command line flag.

It is recommended to use the functions \texttt{print\_errors},
\texttt{print\_heavies} and \texttt{print\_fatal} from the module \texttt{EzyErrors}
to actually phrase the error and only define the layout with the plugin. This is
favorable, because those functions are already internationalized and provide a
common phrasing between the different error printing plugins.

\subsection{The Configuration Directory}
\label{sec:directory}

\easyocaml\ searches for language levels and teachpacks in a designated
configuration directory.

There is a global and a user configuration directory. First, EasyOCaml 
searches the user then the global configuration directory.  Here's how 
the global configuration directory is determined (in descending 
preference):

\begin{enumerate}
    \item Environment variable \texttt{EASYOCAML\_GLOBAL\_DIR}
    \item Compile-time option
\end{enumerate}

Here's how the user configuration directory is determined (in descending 
preference):

\begin{enumerate}
    \item Environment variable \texttt{EASYOCAML\_USER\_DIR}
    \item \texttt{\$HOME/.easyocaml}
\end{enumerate}

\easyocaml's configuration directory must have the following structure:

\begin{verbatim}
 lang-levels/level-1
             level-2
             ...
 teachpacks/tp-1
            tp-2
            ...
\end{verbatim}

Each language level and teach pack contains a module \texttt{LANG\_META} 
which is loaded into EasyOCaml.

