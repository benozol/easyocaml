
\section{The user interface (manual)}
\label{sec:manual}

This section describe the extensions of the \texttt{ocaml} and \texttt{ocamlc} programs.

\subsection{Command line parameters}
\label{sec:commandlineflags}

There are some command line flags to control \easyocaml:

\begin{description}
  \item[\texttt{-easy}] This flag enables \easyocaml\ and is obligatory for
    usage of all other command line flags desribed here.
    It enables an alternative type checking algorithm wich gives more
    information on the type errors.
  \item[\texttt{-easyteachpack <teachpack>}] Enables a teachpack named \texttt{<teachpack>}.
\end{description}

\subsection{Environment variables}

\begin{description}
  \item[\texttt{EASYOCAML\_ENUM\_TIMEOUT}] \label{man:timeout}
    The real value controls the maximal amount of
    time \easyocaml\ may use to enumerate type errors (note, that the underlying algorithm
    has exponential time consumptions).
\end{description}


\subsection{The EasyOCaml directory}

\easyocaml\ searches for language levels and teachpacks in a designated
configuration directory.

There is a global and a user configuration directory. First, EasyOCaml 
searches the user then the global configuration directory.  Here's how 
the global configuration directory is determined (in descending 
preference):

\begin{enumerate}
    \item Environment variable EASYOCAML\_GLOBAL\_DIR
    \item Compile-time option
\end{enumerate}

Here's how the user configuration directory is determined (in descending 
preference):

\begin{enumerate}
    \item Environment variable EASYOCAML\_USER\_DIR
    \item \$HOME/.easyocaml
\end{enumerate}

\easyocaml's configuration directory must have the following structure:

\begin{verbatim}
 language-levels/level-1
                 level-2
                 ...
 teachpacks/tp-1
            tp-2
            ...
\end{verbatim}

Each language level and teach pack contains a module \texttt{LANG\_META} 
which is loaded into EasyOCaml.

