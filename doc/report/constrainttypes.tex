
\section{Constraint Based Type Inference}
\label{sec:typeinfer}

The type inference currently used by \ocaml\ has the algorithm $\mathcal W$ by
\citet{damasmilner} at its core.
Although very efficient for most programs and broadly extended to \ocaml's 
requirements, it lacks sort of a memory:
Inferring the type of a variable is done by accumulating (unifying) information
on its usages while traversing the abstract syntax tree (AST).
Broadly spoken, a type constructor clash is detected when the usage just
inspected contradicts the information collected so far.
Therefore, \ocaml's type checker cannot report any contextual reasons for a type
error, but it reports only the location where the \emph{error became obvious} to
the type checker.
Thus much work while debugging type errors in \ocaml\ comprises of manually
searching for other usages of the mis-typed variable in the program which might
have lead to the type constructor clash.

This section first gives a loose description of \citeauthor{haackwells04}'s type
inference algorithm which covers the very problems just mentioned
---see their \citeyear{haackwells04} paper for a rigid explanation.
It explains the extensions we made for \easyocaml\ afterwards.

\subsection{Haack \& Wells's Type Checking Algorithm}

\citet{haackwells04} describe an algorithm which exceeds algorithm $\mathcal W$
in two ways:
First, every type error report contains information on exactly those
locations in the program which are essential to the error, by means of dropping
one of them would vanish the error.
Second, it is able to report all type errors in a program at once (whilst
locations which are involved in several type errors are most notable the source
of the errors, by the way).

In a sense, the original algorithm $\mathcal W$ deals with two things at once
while traversing the AST: It generates information on the types of the
current variables and unifies it with existing type information anon.
\citeauthor{haackwells04}' algorithm works in some sense by separating those
steps.

During \emph{constraint generation} every node of the AST is first annotated
with a type variable. While traversing the AST, information on those type
variables is collected from the usage of each node. This information is stored
as a set of constraints on the type variables.
The intention is the following: If the constraints are unifiable, the resulting
substitution represents a valid typing of the program with respect to the type
variables of the nodes.
Otherwise, the program contains at least one type error.

But in case of a constraint conflict, the collected type information is still
available as a set of constraints and enables the algorithm to reexamine the
errors in a second stage of \emph{error enumeration and minimization}:
Error enumeration targets to find as many type errors as possible.
This is basically done by systematically removing constraints grounded at one
program location from the constraint set and running unification again such
that the conflict continues to exist.
\citet{haackwells04} also present an iterative version of this algorithm which
is implemented in \easyocaml. Although it avoids repeated computation of the
same errors over and over again, error enumeration has nevertheless exponential
time consumptions.
Thus error enumeration is delimited in \easyocaml\ to a given time amount which
can be specified by an environment variable (see section \ref{man:timeout}).

The result of error enumeration is a set of errors, each represented as a
complete set of locations whose nodes in the AST have contributed to the error
(\emph{complete} in being a superset of the locations which actually caused the
type error).
By application of error minimization to each error, the algorithm further
guarantees \emph{minimality} of the reported errors, in the sense that
removing a single constraint would vanish the constraint conflict, i.e.\ the
error itself.  So, the reported type error contains exactly those locations of
the program which have lead to the error.

Contrary to this approach, other type checkers report whole subtrees of the
program as the reason of an error. This often yields to non-minimality of the
reported error as the subtree might contain locations which do not contribute
to the error. To correctly display only the minimal (and complete) set of
program points, \citeauthor{haackwells04} use \emph{error slices}: All subtrees
of the program, which do not contribute to the error are pruned and replaced by
ellipsises, leaving only a scaffold of the program consisting of the blamed
program points.

In addition to type errors, \citeauthor{haackwells04}' technique also enables
the type checker to collect all unbound variables in the program.  Their types
are assumed as free type variables during type inference to avoid an artificial
type error. But they are reported after unifying the constraints or with the
type errors after error enumeration.


\subsection{Extensions for EasyOCaml}
\label{sec:extending}

\citet{haackwells04} describe the constraint generation rules for
\textsl{MiniML}, a subset of the \textsl{ML} language supporting only
variables, infix operations, functional abstraction, application and local
polymorphic variable bindings.
This is good to describe the algorithms involved, but we had to extend it to
the language \camlm---as loosely described in section \ref{sec:language} and
more formally in section \ref{sec:grammar}---and its much richer type system.
This section will describe the constraint generation rules for \easyocaml\ by
example here, section \ref{sec:rules} exposes the complete set of rules.

In the following, $\Delta$ always denotes an environment (store) for current
bindings of variables, record fields and variant constructors anon, accessible
by $\Delta\onident$, $\Delta\onvar$, $\Delta\onrecord$ and $\Delta\onvar$
respectively.

To capture the possibility to \emph{declare} values and types, constraint
generation rules for structure items have the form

\[ \Delta;\ strit \Downarrow_s \langle \Delta',\ C,\ u\rangle.\]

\noindent where $\Delta$ denotes the environment which contains declarations in
the program so far and $strit$ denotes the current structure item.
$\Delta'$ denotes the environment $\Delta$ extended by declarations in $strit$
and $C$ is the set of constraints collected in $strit$.
$u$ is a set of errors in $strit$ which are described in more detail in section
\ref{sec:easyerrors}.
Those declarations, constraints and errors are accumulated while traversing the
program's structure items.

Here is the rule for the declaration of a variant type:
\[\inferrule[Variant Decl]
{\Delta' = \Delta\onvar[t \mapsto \{ \langle K_1, ty_1\rangle^{l_1}, \dots, \langle K_n, ty_n\rangle^{l_n}\}^l]}
{\styjudge \Delta {(\code{type}\ t = K_1\ \code{of}^{l_1}\ ty_1\ \code | \dots \code | K_n\ \code{of}^{l_n}\ ty_n)^l} {\Delta'} \emptyset \emptyset} \]
It just extends the current environment $\Delta$ with information on the
variant constructors $K_1$ to $K_n$ with the given types.
Note, that the locations are stored to make a reference on the type declaration
in case of a typing error related to one of those variant constructors.
This information is available in $\Delta\onvar$ during constraint generation of
the subsequent program code and can be used to assert the proper argument and
result types of variant constructors.

Constraint generation rules for expressions are a better example for the
process of accumulating constraint sets.
As a simple starting point, we will discuss the rule for \texttt{if} expressions
without an \texttt{else} branch here.
\ocaml\ provides the test expression to be of type \type{bool}, the expression
in the branch of type \type{unit} and the whole expression of type \type{unit},
too.
This is implemented in the following rule:
\[\inferrule[If-then]
{\etyjudge \Delta {lexp_1} {ty_1} {C_1} {u_1} \\
 \etyjudge \Delta {lexp_2} {ty_2} {C_2} {u_2} \\
 C_0 = \{ ty_1 \xlongequal l \type{bool},\ ty_2 \xlongequal l \type{unit},\ a \xlongequal l \type{unit} \} \\
 a \text{ fresh}}
{\etyjudge \Delta {(\code{if}\ lexp_1\ \code{then}\ lexp_2)^l} a C {u_1 \cup u_2}} \]
Constraint generation is applied here to $lexp_1$ (and $exp_2$ respectively),
resulting  in the type $ty_1$ (and $ty_2$) which is in fact a type variable
constrained to the result type of $lexp_1$ in $C_1$ (and $lexp_2$ in $C_2$).
The rule generates three additional constraints.  The first one, $ty_1
\xlongequal l \type{bool}$, asserts the result type $ty_1$ of expression
$lexp_1$ to be of type $\type{bool}$.
The second, $ty_2 \xlongequal l \type{unit}$ asserts the result type $ty_2$ of
the expression $lexp_1$ to be of type $\type{unit}$.
The third one, $ty_3 \xlongequal l \type{unit}$, asserts a freshly generated
type variable $ty_3$, which represents the type of the whole conditional
expression, to be of type $\type{unit}$.

Note, that all new constraints are annotated with the location $l$ of the
overall expression. This facilitates the accusation of it in case of a type
error resulting from a conflict with one of the constraints special to this
conditional expression.
The constraint generation results in the generated type variable $ty_3$ and the
union of all occurring constraint and error sets.

A more particular feature of \easyocaml\ are type annotations of the form
$(lexp:ct)$.  Special considerations are necessary for them:
\ocaml's current type checker ignores the type annotations during unification by
using the expression $lexp$'s inferred type for further type checking and only
checks its validity as to $ct$ afterwards.  So type inference and error
reporting makes no use of the type annotations itself.

In contrast, \easyocaml\ assumes the expression to have the denoted type while
type checking, type checks the expression isolated and tests the validity
afterwards, by proving that the annotated type is a subtype of the inferred
type.
\[\inferrule[Type-Annot]{
\etyjudge \Delta {lexp} {ty} {C_0} u \\
C_1 = \{a \xlongequal l ty',\ ty \succcurlyeq^l ct\} \\
a \text{ fresh} \\
ty' \text{ is a fresh instance of } ct \\
}{
  \etyjudge{\Delta}{(lexp : ct)^l}{a} {C_0 \cup C_1} {u}
}\]
A freshly generated type variable $a$ represents the type of the overall
expression. It is constrained to a fresh instance of the annotated type. The
constrain $ty \succcurlyeq^l ct$ captures the validity of the annotated
expression's type $ty$ with respect to the annotated type $ct$ to catch up on
after constraint unification.
That way, \easyocaml\ assumes the programmer's annotation to be valid and
meaningful during type inference of the context of the overall expression and
checks for a contradiction to the expression's type against the annotation
afterwards.

Constraint generation rules for patterns have the form
\[\Delta;\ pat \Downarrow_p \langle ty,\ C,\ b\rangle\]
and provide the type of the values that can be matched with it as well as a set
of constraints on it and all nested patterns. But contrary to constraint
generation rules for expressions, all occurring variables are provided as bound
to given parts of the matched value.
$b$ is a mapping of variables in the pattern to their according type variable
constrained by $C$.
Note that constraint generation rules for compound patterns like
\[\inferrule[Tuple]
{
  \ptyjudge \Delta {pat_i} {ty_i} {C_i} {b_i} \text{ for } i = 1,\dots,n \\
  C_0 = \{ a \xlongequal l (ty_1, \dots, ty_n) \} \\
  \dom(b_i) \cap \dom(b_j) = \emptyset \text{ for } i \neq j \\
  a \text{ fresh}
  }
{\ptyjudge \Delta {(pat_1\code ,\ \dots\code ,\ pat_n)^l} a {\bigcup_{i=0}^n
C_i} {\bigcup_{i=1}^n b_i}}\]
assert the sets of bound variables for sub-patterns to be mutually exclusive.

Additionally, there is a kind of auxiliary constraint generation rules:
\[\Delta;\ pat\ \code{->}\ lexp {\color{Gray}\ |\ rules} \Downarrow_r \langle
ty_p,\ ty_e,\ C,\ u\rangle.\]
This covers the mapping of patterns to expressions as used in value matching,
functional abstraction with the keyword \texttt{function} and exception catching
to deal with different variants of a value.
The rule arranges the bound variables of the patterns to be available while
generating constraints on the right hand side expressions and further generates
constraints on the equality of the pattern's as well as the expression's types.

\easyocaml's constraint generation does a second task anon: It checks the
validity of the variables, record constructors, field accesses and type
constructions. The handling of those and other errors are described in the
next section.
