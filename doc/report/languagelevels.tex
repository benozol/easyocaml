
\section{Language Levels and Teachpacks}
\label{sec:teachpacks}

Language levels and teach packs are didactical tools to simplify a language and
its handling for beginners. They were introduced by \drscheme\
\citep{Felleisen98thedrscheme}  for two reasons: First, their Scheme language is
split into a ``tower of languages''---that is different levels of syntactical
and semantical richness. And language levels are used to specify the available
language. Second, language levels and teach packs can provide arbitrary code to
the user in a simple and encapsulated way. In contrast, teach packs contain only
additional code and can extend a currently active language level.

That way language levels and teachpack are a mean to adapt the languale to the
knowledge of students at a dedicated state of the class: Reasonable levels of
the tower of languages are e.g.\ a first-order functional language, a
higher-order functional language or a language with imperative features like
mutable date.

To use the full parser and to check for the syntactical restraints afterwards
would facititate syntax errors regarding syntactical categories not included in
the current language level---very confusing for beginners. 
\easyocaml, however, directly manipulates the parser to the requirements of the
language level such that the parser itself does not even ``know'' about the
syntactical categories not part of the language level and so does not report
errors regarding to them.

\easyocaml\ gives further fine grained control over the available syntax.
Every option for non-terminal nodes in the grammar (structure items, type
declarations  expressions, patterns) can be switched on and off. This can be
done even independently for different usages, e.g. the patterns in value
matching\ttfootnote{match \dots\ with \emph{pat} -> \dots} can be configured in
a different way than the patterns in functional
abstractions\ttfootnote{function \emph{pat} -> \dots}. This is implemented by
deleting the minimal common disabled set of options from the \camlpf\ grammar.
At a second stage (namely while transforming the \ocaml\ AST into
\easyocaml's), all remaining restrictions are examined.

As mentioned, language levels and teach packs can be used to provide code at
starting time of the compiler and toplevel loop in a simple manner. They contain
a list of modules which ought to be availble to the user each with an annotation
if they are opened at start-up time. That way the predefined module
\texttt{Pervasives} can be extended to any needs.
Section \ref{sec:manual} shows how to define your own language level or teach
pack.

The user can specify which language level and teach pack to use by the 
\texttt{-easylevel} and \texttt{-easyteachpack} command line parameter 
respectively. \easyocaml\ then searches for it in a dedicated directory as
described in section \ref{sec:directory}.

