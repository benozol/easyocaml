
\section{Objectives and Introduction}
\label{sec:intro}

\label{hd001}
\textsl{Objective Caml} \citep{leroy2008} is a programming language which unifies 
functional, imperative and object oriented concepts in a ML-like 
language with a powerful and sound type system.  Its main implementation%
\footnote{\url{http://caml.inria.fr}} ships with a platform independent byte code
compiler and an efficient machine code compiler and there are a lot of
libraries, which make it a great multi purpose programming language.

But up to now, \ocaml\ is not a language well suited for \emph{learning} and
\emph{teaching programming}:
It has a very rich type system, but type errors are reported only with few
information on the underlying reasons.
Some practice is necessary to manage these.
On the other hand, OCaml comes with tools (e.g.  \citeauthor{stolpmann}'s
\emph{findlib}) which make it easy to handle libraries for developers, but it
lacks a fool-proof system to use primed code in programming lessons.

The objectives of \easyocaml\ in large are to make OCaml a programming language
better suited for beginners and for teaching programming. We achieve this by

\begin{itemize}
    \item improving \ocaml's error messages by providing a modified 
        parser and a new type checker.
    \item equipping \ocaml\ with an infrastructure to make it 
        adaptable for teaching programming in means of restricting the
        supported features of the language and providing code and the 
        startup environment in a simple way of distribution (language 
        levels).
    \item integrating all that into \ocaml's original toploop and 
        compiler system to take advantage of existing libraries and 
        \ocaml's code generation facilities.
\end{itemize}
The project is hosted at \url{http://easyocaml.forge.ocamlcore.org}.
The website comes with an online demo of \easyocaml's type inference system and
some languge levels.

\subsection*{Similar Projects}

There are some projects which heavily influenced our work:

\citet{haackwells04} described and implemented a technique to produce more
descriptive type error messages in a subset of SML.  
Their work is seminal for constraint based type checking with attention 
on good error reporting.

\program{Helium} \citep{helium-hw03} is a system for teaching programming 
in Haskell. In a similar manner, type checking is done via constraint 
solving. Furthermore, it features detailed error messages including 
hints how to fix errors based on certain heuristics.

Finally, \program{DrScheme} \citep{Felleisen98thedrscheme} is a programming 
environment for the Scheme language which is built for teaching
programming.  Beside the integration of the editor and the REPL as well as an
easy-to-use debugger (\emph{stepper}), \program{DrScheme} has first introduced
the concept of language levels and teach packs to restrict the syntax and
broaden functionality especially for exercises.

Here is the way \easyocaml\ puts together those elements: The type checker is an
extension of \citet{haackwells04}'s  algorithm to a significant subset of
\ocaml. It is fairly similar to \program{DrScheme} in scope, as it features languge levels
and teach packs in the same way as \program{DrScheme} does for usage in programming
courses. Although not yet implemented, \easyocaml\ can build the basis for a
tutoring system like \program{Helium} by providing enough information to give
advises in case of an error.


\subsection*{Goals Of This Report}

This report is split in a rather ``narrative'' part \ref{part:concepts}%
---answering the question: \emph{what?}---and a more formal part \ref{part:implem}
which acts as a reference for users and developers---answering the question:
\emph{how?}.

The report has several goals:
First to present \easyocaml\ and the concepts used in and developed for it
(target audience: the users). Beside general informations in
section~\ref{sec:general} it covers the implemented type inference algorithm in
section~\ref{sec:typeinfer}, error handling in section~\ref{sec:errors} and
language levels in section~\ref{sec:teachpacks}.
Then, it describes the usage of the implemented programs \texttt{ecaml} and
\texttt{ecamlc} in section~\ref{sec:manual}.
The grammar of \easyocaml's language is given in rail road diagrams in section
\ref{sec:grammar}.
Afterwards formal foundations for the type checker are given the
section~\ref{sec:rules} (target audience: the interested readers).
Finally, section \ref{impldets} combines the concepts of the first sections with
the actual implementation and describes the scheme of the code (target audience:
the developers).

