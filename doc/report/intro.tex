
\section{Objectives and Introduction}
\label{sec:intro}

\label{hd001}
\textsl{Objective Caml} \citep{leroy2008} is a programming language which unifies 
functional, imperative and object oriented concepts in a ML-like 
language with a powerful and sound type system.  Its main implementation 
(\url{http://caml.inria.fr}) ships with a platform independent byte code
compiler and an efficient machine code compiler and there are a lot of
libraries, which make it a great multi purpose programming language.

But up to now, \ocaml\ is not a language well suited for \emph{learning} and
\emph{teaching programming}:
It has a very rich type system, but type errors are reported only with few
information on the underlying reasons.
Some practice is necessary to manage these.
On the other hand, OCaml comes with tools (e.g.  \citeauthor{stolpmann}'s
\emph{findlib}) which make it easy to handle libraries for developers, but it
lacks a fool-proof system to use primed code in programming lessons.

The objectives of this work are, in large, to make OCaml a
programming language better suited for beginners and to teach programming. We
achieve this by

\begin{itemize}
    \item improving \ocaml's error messages by providing a modified 
        parser and a new type checker.
    \item equipping \ocaml\ with an infrastructure to make it 
        adaptable for teaching programming, in means of restricting the 
        supported features of the language, or providing code and the 
        startup environment in a simple way of distribution (language 
        levels).
    \item integrating all that into \ocaml's original toploop and 
        compiler system to take advantage of existing libraries and 
        \ocaml's code generation facilities.
\end{itemize}
The project is hosted at
\url{http://easyocaml.forge.ocamlcore.org}  where an online demo of \easyocaml's
type inference system and languge levels is available, too.

\subsection*{Similar Projects}

There are some projects which heavily influenced our work:

\citet{haackwells04} have described and implemented a technique 
to produce more descriptive type error messages in a subset of SML.  
Their work is seminal for constraint based type checking with attention 
on good error reporting and builds the foundation for \easyocaml's type 
checker.

\emph{Helium} \citep{helium-hw03} is a system for teaching programming 
in Haskell. In a similar manner, type checking is done via constraint 
solving. Furthermore, it features detailed error messages including 
hints how to fix errors based on certain heuristics.

Finally, \emph{DrScheme} \citep{Felleisen98thedrscheme} is a programming 
environment for the Scheme language which is build for teaching 
programming.  Apart from an integration of the editor and the REPL as well as a
easy to use debuggen (\emph{stepper}), it has introduced the concept of language
levels and teach packs to restrict the syntax and broaden functionality
especially for exercises.

\easyocaml\ uses and combines ideas of all these projects for the attempt to
make \ocaml\ better suited for learning and teaching programming. \more


\subsection*{Goals of this report}

This report has four goals:
First to present \easyocaml\ and the concepts used in and developed for it
(target audience: the users) through section~2--5.
Then, it describes the usage of the programs \texttt{ecaml} and \texttt{ecamlc}
for usage in section~\ref{sec:manual}.
Afterwards it combines the concepts with the actual implementation and
describes its architecture in code (target audience: the developers) in
section~\ref{impldets}.
Finally, formal foundations for the type checker are given the section~\ref{sec:rules} (target
audience: the interested readers).

This report is also split in a less formal part \ref{part:concepts} (answering
the question: \emph{what?}) and a more formal part \ref{part:implem}
which acts as a reference for users and developers (answering the question:
\emph{how?}).
