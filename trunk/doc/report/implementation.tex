
\section{Details of the Implementation}
\label{impldets}

\subsection{Dependency Graph of EasyOCaml's modules}

\includegraphics[width=\textwidth]{dot_deps}

\subsection{Short Descriptions of the Modules}

\subsubsection{Utilities and Miscellaneous}

Two rather independent modules for code used in EasyOCaml

\begin{description}
    \item[EzyUtils] Functionality which is not specific to 
        EasyOCaml, but extends the standard library (String, Set, Map). 
        It contains also copies code from existing Libraries (from Core: 
        Option, Monad, T2, T3, T4, such that EasyOCaml adds no 
        dependencies at bootstrap time) and new code for Logging and 
        some more (lexical comparison, tools on functions).
    \item[EzyMisc] EasyOCaml-specific code which is used at 
        different locations in the project.
    \item[EzyOcamlmodules] Extensions of the modules from the 
        standard OCaml system (e.g. Location, Path, Longident, Types, 
        \ldots{}) as well as sets and maps of these.
\end{description}

The rest of the modules contains the code for the EasyOCaml 
implementation:


\subsubsection{Error Reporting}

EasyOCaml offers sophisticated facilities to represent errors, to allow 
as detailed error reporting as possible.  Furthermore, new error 
reporting plugins can be registered.

\begin{description}
    \item[EzyErrorReportUtils] Code for type error slicing 
      (described in \cite{haackwells04}), i.e. slicing an AST to only contain
      nodes from locations given in a set, substituting the rest with ellipsis.
    \item[EzyErrors] Representation (types) of errors which can 
        occur in EasyOCaml, functions for pretty printing errors as well 
        as functions to register error reporting plugins.
\end{description}


\subsubsection{Teachpacks and Language Levels}

\begin{description}
    \item[EzyConfig] Constants of the teach pack system (e.g. the 
        name of the module describing the teach pack or language level) 
        and functions to find a teach pack or language level in the file 
        system.
    \item[EzyDynload] Superset of functionality for loading teach 
        packs and language levels (used by \texttt{EzyLang})
    \item[EzyLang] Functions for loading language levels and teach 
        packs (used by \texttt{EzySetup})
    \item[EzyTeachpack] Shortcut to \texttt{EzyFeatures} and 
        registering of the teach pack.  Actual teach packs should only 
        need to link against this module.
    \item[EzyLangLevel] Shortcut to \texttt{EzyFeatures} and 
        registering of the language level.  Actual language levels 
        should only need to link against this module.
    \item[EzySetup] Process command line flags regarding language 
        levels and teach packs and provide the actual setup of features, 
        modules, included directories and object files given by teach 
        packs and language levels to other parts of EasyOCaml.
\end{description}


\subsubsection{Abstract Syntax Tree}

The following modules contain representation, manipulation, parsing and 
restrictions on EasyOCaml's AST.

\begin{description}
    \item[EzyFeatures] In EasyOCaml, the available syntax can be 
        restricted.  This module contains types to describe these 
        restrictions and some functions to generate defaults (i.e. 
        settings where everything is forbidden or allowed).
    \item[EzyAsttypes] Adaption of Asttypes from the standard OCaml 
        system.
    \item[EzyAst] Representation of the AST in EasyOCaml.  Each node 
        is parametrized on some data it contains.  This is \texttt{unit} 
        for a parsed tree and typing information (mainly the type 
        variable) for a parsed tree after constraint generation. 
        Furthermore, each syntactic category can be an ellipsis which 
        is only used in type error slicing. 
    \item[EzyCamlgrammar] The EasyOCaml Parser as a Camlp4 extension 
        of \texttt{Camlp4OCamlParser}.  It just deletes some entries in 
        the latter partially depending on the features specified by the language
        level.
    \item[EzyEnrichedAst] This module directly belongs to 
        \texttt{EzyAst} but we had to outsource it because of module 
        dependencies between \texttt{EzyErrors}.  It contains
        \begin{itemize}
            \item definitions of the AST after constraint generation 
                import functions from
            \item OCaml's standard Parsetree respecting given 
                restrictions from \texttt{EzyFeatures}
            \item comparison of two ASTs which is used to compare 
                OCaml's typing and EasyOCaml's typing afterwards
        \end{itemize}
\end{description}


\subsubsection{Type Constraints}

\begin{description}
    \item[EzyTypingCoreTypes] Contains base types for the 
        constraints and their generation, closely related to the data 
        described in \cite{haackwells04} (type variables, types, type 
        substitutions, intersection types, type environments)
    \item[EzyConstraints] Here are constraints annotated with only 
        one location (\texttt{AtConstr.t}) and constraints with sets of 
        locations (\texttt{Constr.t}) defined, as well as set and maps of those.
        Furthermore a derived environment as described in \cite{haackwells04}
        is defined.
    \item[EzyGenerate] There is a function for every syntactic 
        category to generate constraints and/or errors.
\end{description}

\subsubsection{Typing}

\begin{description}
    \item[EzyTyping] Unification of a set of constraints which yield a 
        substitution on the variables and error enumeration and 
        minimization as described by \cite{haackwells04}.  It furthermore 
        contains convenient typing functions for structures which are used in
        the compiler and toplevel.
    \item[EzyEnv] The \texttt{EzyEnv.t} is the typing environment 
        for EasyOCaml. Information on declared types and types of local 
        and global variables is hold.  It is build up while constraint 
        generation (\texttt{EzyGenerate}) in combination with the type 
        variable substitution resulting from \texttt{EzyTyping.solve}.
\end{description}

