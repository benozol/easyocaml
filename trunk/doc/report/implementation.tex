
\section{Details of the Implementation}
\label{impldets}

\subsection{Outline of EasyOCaml's Pipeline}
\label{hd005001}
Here is a rough outline of EasyOCaml's pipeline which is quite similar 
for both the compiler and the toploop:

\begin{enumerate}
    \item \label{steps} Command line flags are evaluated to check 
        the ``\texttt{-easy}'' flag and an error printer and possibly 
        load a teach pack and/or language levels.
    \item \texttt{EzyCamlgrammar} parses the AST from the input, 
        possibly respecting restrictions from the language level which 
        yields an \texttt{EzyAst.imported\_structure}.
    \item \texttt{EzyGenerate} generates constraints from the AST 
        (involving type information from the default environment and 
        modules loaded by command line or teach packs/language levels).  
        This yields a quadruple 
        \texttt{generated\_structure~*~AtConstrSet.t~*~PostProcess.t~*~EzyEnv.t} 
        where
        \begin{description}
            \item[\texttt{generated\_structure}] is the AST 
                annotated with type variables and unique identifiers.
            \item[\texttt{AtConstrSet.t}] is a set of constraints on 
                the type variables in the AST.
            \item[\texttt{PostProcess.t}] is build gradually during 
                constraint generation and contains sets of different 
                types of errors (from \texttt{EzyErrors}) as well as 
                checks which can only processed after constraint 
                unification (i.e. type annotations)
            \item[\texttt{EzyEnv.t}] is used to keep track of local 
                variables and after constraint generation contains 
                information on the global types and values of the 
                program.
        \end{description}
    \item \texttt{EzyTyping.solve} tries to solve the generated 
        constrains.  If solving succeeds, the program is typed by a 
        substitution on the variables in the generated AST and the 
        environment and contains type errors otherwise.
    \item The last step, reimporting the 
        \texttt{EzyEnrichedAst.generated\_structure} with typing 
        information given by type substitution to OCaml's original typed 
        tree is not yet done.  We type the code again with OCaml's 
        original type checker and compare the result to verify its 
        correctness.
\end{enumerate}

The goal of this section is to describe roughly the modules implemented 
for EasyOCaml and locate functions for the EasyOCaml's steps in 
~\ref{steps}.


\subsection{Utilities and Miscellaneous}
\label{hd005002}
Two rather independent modules for code used in EasyOCaml

\begin{description}
    \item[EzyUtils] Functionality which is not specific to 
        EasyOCaml, but extends the standard library (String, Set, Map). 
        It contains also code copies from existing Libraries (from Core: 
        Option, Monad, T2, T3, T4, such that EasyOCaml adds no 
        dependencies at bootstrap time) and new code for Logging and 
        some more (lexical comparison, tools on functions).
    \item[EzyMisc] EasyOCaml-specific code which is used at 
        different locations in the project.
    \item[EzyOcamlmodules] Extensions of the modules from the 
        standard OCaml system (e.g. Location, Path, Longident, Types, 
        \ldots{}) as well as sets and maps over these.
\end{description}

The rest of the modules contains the code for the EasyOCaml 
implementation:


\subsection{Error Reporting}
\label{hd005003}
EasyOCaml offers sophisticated facilities to represent errors, to allow 
as detailed error reporting as possible.  Furthermore, new error 
reporting plugins can be registered.

\begin{description}
    \item[EzyErrorReportUtils] Code for type error slicing 
        (described in Haack \& Wells), i.e. slicing an AST to only 
        contain nodes from locations given in a set, substituting the 
        rest with ellipses.
    \item[EzyErrors] Representation (types) of errors which can 
        occur in EasyOCaml, functions for pretty printing errors as well 
        as functions for error reporting plugins to register themselves.
\end{description}


\subsection{Teachpacks and Language Levels}
\label{hd005004}
\begin{description}
    \item[EzyConfig] Constants of the teach pack system (e.g. the 
        name of the module describing the teach pack or language level) 
        and functions to find a teach pack or language level in the file 
        system.
    \item[EzyDynload] Superset of functionality for loading teach 
        packs and language levels (used by \texttt{EzyLang})
    \item[EzyLang] Functions for loading language levels and teach 
        packs (used by \texttt{EzySetup})
    \item[EzyTeachpack] Shortcut to \texttt{EzyFeatures} and 
        registering of the teach pack.  Actual teach packs should only 
        need to link against this module.
    \item[EzyLangLevel] Shortcut to \texttt{EzyFeatures} and 
        registering of the language level.  Actual language levels 
        should only need to link against this module.
    \item[EzySetup] Process command line flags regarding language 
        levels and teach packs and provide the actual setup of features, 
        modules, included directories and object files given by teach 
        packs and language levels to other parts of EasyOCaml.
\end{description}


\subsection{Abstract Syntax Tree}
\label{hd005005}
The following modules contain representation, manipulation, parsing and 
restrictions on EasyOCaml's AST.

\begin{description}
    \item[EzyFeatures] In EasyOCaml, the available syntax can be 
        restricted.  This module contains types to describe these 
        restrictions and some functions to generate defaults (i.e. 
        settings where everything is forbidden or allowed).
    \item[EzyAsttypes] Adaption of Asttypes from the standard OCaml 
        system.
    \item[EzyAst] Representation of the AST in EasyOCaml.  Each node 
        is parametrized on some data it contains.  This is \texttt{unit} 
        for a parsed tree and typing information (mainly the type 
        variable) for a parsed tree after constraint generation. 
        Furthermore, each syntactic category can be some ``dots'' which 
        is only used in type error slicing.
    \item[EzyCamlgrammar] The EasyOCaml Parser as a Camlp4 extension 
        of \texttt{Camlp4OCamlParser}.  It just deletes some entries in 
        the latter (partially depending on the given features.
    \item[EzyEnrichedAst] This module directly belongs to 
        \texttt{EzyAst} but we had to outsource it because of module 
        dependencies between \texttt{EzyErrors}.  It contains
        \begin{itemize}
            \item definitions of the AST after constraint generation 
                import functions from
            \item OCaml's standard Parsetree respecting given 
                restrictions from \texttt{EzyFeatures}
            \item comparison of two ASTs which is used to compare 
                OCaml's typing and EasyOCaml's typing afterwards
        \end{itemize}
\end{description}


\subsection{Type Constraints}
\label{hd005006}
\begin{description}
    \item[EzyTypingCoreTypes] Contains base types for the 
        constraints and their generation, closely related to the data 
        described in Haack \& Wells (type variables, types, type 
        substitutions, intersection types, type environments)
    \item[EzyConstraints] Here are constraints annotated with only 
        one location (\texttt{AtConstr.t}) and constraints with sets of 
        locations (Constr.t) defined, as well as set and maps of those.  
        Furthermore a derived environment as described in Haack \& Wells 
        is defined.
    \item[EzyGenerate] There is a function for every syntactic 
        category to generate constraints and/or errors.
\end{description}

\subsection{Typing}
\label{hd005007}
\begin{description}
    \item[EzyTyping] Unification of constraint set which yield a 
        substitution on the variables and error enumeration and 
        minimization as described by Haack \& Wells.  It furthermore 
        contains the typing functions for structures which are used in 
        the compiler and toplevel.
    \item[EzyEnv] The \texttt{EzyEnv.t} is the typing environment 
        for EasyOCaml. Information on declared types and types of local 
        and global variables is hold.  It is build up while constraint 
        generation (\texttt{EzyGenerate}) in combination with the type 
        variable substitution resulting from \texttt{EzyTyping.solve}.
\end{description}

