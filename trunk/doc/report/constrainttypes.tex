
\section{Constraint Based Type Inference}
\label{sec:typeinfer}

The type inference currently used by \ocaml\ has the algorithm $\mathcal W$ by
\citet{damasmilner} at its core.
Although very efficient for most programs and broadly extended to \ocaml's 
requirements, it lacks sort of a memory:
Infering the type of a variable is done by accumulating (unifying) information
on its usages while traversing the abstract syntax tree (AST).
Broadly spoken, a type constructor clash is detected as the usage just
inspected contradicts the information collected so far.
Therefore, \ocaml's type checker cannot report any contextual reasons for a type
error but it reports only the location where the error became obvious to the
type checker.
Much work while debugging type errors in \ocaml\ comprises thus of manually
searching for other usages of the mis-typed variable in the program which might
have lead to the type constructor clash.

I will first loosely describe the algorithm by \citeauthor{haackwells04}---see
their \citeyear{haackwells04} paper for a rigid explanation---and then explain
the extensions we made for \easyocaml.


\subsection{Haack \& Wells's Type Checking Algorithm}

\citet{haackwells04} describe an algorithm which exceeds algorithm $\mathcal W$
in two ways:
Firstly, every type error report contains information on exactly those
locations in the program which are essential to the error, by means of dropping
one of them would vanish the error.
Secondly, it is able to report all type errors in a program at once (whilst
locations which are involved in several type errors are most notable the source
of the errors, by the way).

In a sense, algorithm $\mathcal W$ does two things at once while traversing the 
AST: It generates information on the types of the variables and unifies it with
existing type information anon.  \citeauthor{haackwells04}' algorithm works in
some sense by separating these steps.

During \emph{constraint generation} every node of the AST is\wha\ annotated with
a type variable. While traversing the AST, information on those type variables
is collected from the usage of each node. This information is stored as a set of
constraints on the type variables.
The intention is the following: If the constraints are unifiable, the resulting
substitution represents a valid typing of the program with respect to the type
variables of the nodes.
Otherwise, the program has at least one type error.

But in case of a constraint conflict, the collected type information is still
available as a set of constraints and enables the algorithm to reexamine the
errors in a second stage of \emph{error enumeration} and \emph{minimization}:
Error enumeration is basically done by systematically removing constraints
grounded at one program location from the constraint set and running unification
again.
\citet{haackwells04} also present an iterative version of this algorithm which
is implemented in \easyocaml. Although it avoids recomputation of the same
errors over and over again, error enumeration has nevertheless exponential time
consumptions. Thus error enumeration is delimited in \easyocaml\ to a given time
amount which can be specified by an environment variable (see section
\ref{man:timeout}).

The result of error enumeration is a set of errors, each represented as a
complete set of locations whose nodes in the AST have contributed to the error
(\emph{complete} in being a superset of the locations which caused the type
error). By application of error minimization on each error, the algorithm
further guarantees \emph{minimality} of the reported errors, in the sense that
removing the constraints annotated with a single location would vanish the error
itself.
So, the reported type error contains exactly those locations of the program
which lead to the error.

Contrary to this approach, other type checkers report whole subtrees of the
program as the reason of an error. This often yields to non-minimality of the
reported error (as the subtree might contain locations which do not contribute
to the error). To correctly display only the minimal (and complete) set of
progam points, \citeauthor{haackwells04} use \emph{error slices}: All subtrees
of the program, which do not contribute to the error are pruned and replaced by
elipsises, leaving only a scaffold of the program containing the blamed program
points.

In addition to type errors, \citeauthor{haackwells04}' technique also enables the 
type checker to collect all unbound variables in the program.
Their types are assumed as free type variables to avoid a type error, but
reported after unifying the constraints or with the type errors after error
enumeration.


\subsection{Extensions for EasyOCaml}
\label{sec:extending}

\citet{haackwells04} comes with constraint generation rules for
\textsl{MiniML}, a subset of the \textsl{ML} language only supporting
variables, infix operations, functional abstraction, application and local
polymorphic variable bindings.
%\easyocaml\ features a much richer type system, as it facilitates the
%declaration of values, record types, variant types and type synonyms.
This is good to describe the algorithms involved, but we had to extend it to the
language \camlm---as loosely described in section \ref{sec:language} and more
formally in section \ref{sec:grammar}---and \easyocaml's much richer type
system.
I will describe the constraint generation rules for \easyocaml\ by example
here, section \ref{sec:rules} exposes the complete set of rules.

In the following, $\Delta$ always denotes an environment (store) for current
bindings of variables, record fields and variant constructors anon, accessible
by $\Delta\onident$, $\Delta\onvar$, $\Delta\onrecord$ and $\Delta\onvar$
respectively.

To capture the possibility to \emph{declare} values and types, constraint
generation rules for structure items have the form

\[ \Delta;\ strit \Downarrow_s \langle \Delta',\ C,\ u\rangle.\]

\noindent where $\Delta$ denotes the environent which contains declarations in
the program so far and $strit$ denotes the current structure item.
$\Delta'$ denotes the environment $\Delta$ extended by declarations in $strit$
and $C$ is a set of constraints collected in $strit$.
$u$ is a set of errors in $strit$ which are desribed in more detail in section
\ref{sec:easyerrors}.
Those declarations, constraints and errors are accumulated while generating
traversing the program's structure items.

This is the rule for the declaration of a variant type:

\vspace{1em}\centerline{
\inferrule[Variant Decl]
{\Delta' = \Delta\onvar[t \mapsto \{ \langle K_1, ty_1\rangle^{l_1}, \dots, \langle K_n, ty_n\rangle^{l_n}\}^l]}
{\styjudge \Delta {(\code{type}\ t = K_1\ \code{of}^{l_1}\ ty_1\ \code | \dots \code | K_n\ \code{of}^{l_n}\ ty_n)^l} {\Delta'} \emptyset \emptyset}}
\vspace{1em}

\noindent It just extends the current environment $\Delta$ with the variant constructors
$K_1$ to $K_n$ with the given types.
Note, that the locations are stored to make a reference on the type declaration
in case of a typing error related to one of those variant constructors.

Constraint generation rules for expressions are better examples for
accumulating constraint sets.
As a simple starting point, we will discuss the rule for \texttt{if} expressions
without an \texttt{else} branch here.
\ocaml\ provides the test expression to be of type \type{bool}, the expression
in the branch of type \type{unit} and the whole expression of type \type{unit},
too. The rule implements this a follows:

\vspace{1em}\centerline{
\inferrule[If-then]
{\etyjudge \Delta {lexp_1} {ty_1} {C_1} {u_1} \\
 \etyjudge \Delta {lexp_2} {ty_2} {C_2} {u_2} \\
 C_0 = \{ ty_1 \xlongequal l \type{bool},\ ty_2 \xlongequal l \type{unit},\ a \xlongequal l \type{unit} \} \\
 a \text{ fresh}}
{\etyjudge \Delta {(\code{if}\ lexp_1\ \code{then}\ lexp_2)^l} a C {u_1 \cup u_2}}
}\vspace{1em}

Constraint generation is applied here to $lexp_1$ (and $exp_2$ respectively),
resulting  in the type $ty_1$ which is in fact a type variable constraint to
the result type of $lexp_1$ in $C_1$.
The rule generates three additional constraints.  The first one, $ty_1
\xlongequal l \type{bool}$, asserts the result type $ty_1$ of expression
$lexp_1$ to be of type $\type{bool}$.
The second, $ty_2 \xlongequal l \type{unit}$ asserts the result type $ty_2$ of
the expression $lexp_1$ to be of type $\type{unit}$.
The third one, $ty_3 \xlongequal l \type{unit}$, asserts that a freshly
generated type variable $ty_3$, which represents the type of the whole
conditional expression, is of type $\type{unit}$.

Note, that all new constraints are annotated with the location $l$ of the
overall expression. This facilitates the blaming of it in case of a type error
resulting from a conflict with one of the constraints special to this
conditional expression.

The constraint generation results in the generated type variable $ty_3$ and the
union of all occurring constraint and error sets.

Furthermore, \easyocaml\ features type annotations of the form $(lexp:ct)$.
Special considerations are necessary for them:
OCaml's current type checker ignores the type annotations whilst unification by
using the expression $lexp$'s inferred type for further type checking and only
checks its validity as to $ct$ later on.  So type inference and error reporting
makes no use of the type annotations itself.

In contrast, \easyocaml\ assumes the expression to have the denoted type while
type checking, type checks the expression isolated and tests the validity
afterwards, by proving that the annotated type is a subtype of the inferred type.

\vspace{1em}\centerline{
\inferrule[Type-Annot]{
  \etyjudge{\Delta}{lexp}{ty}{C_0}{u} \\
  C_1 = \{a \xlongequal l b,\ b \xlongequal {l'} ty',\ ty \xlongequal l c,\ c \xlongequal {l'} ty',\ a \succcurlyeq^l ct\} \\
  a,\ b,\ c \text{ fresh} \\
  ty' \text{ is a fresh instance of } ct \\
}{
  \etyjudge{\Delta}{(lexp : ct^{l'})^l}{a} {C_0 \cup C_1}{u}
}
}\vspace{1em}

That way, \easyocaml\ assumes the programmer's annotation to be valid and
meaningful during type inference of the context of $lexp$ and checks for
contradiction to the expression's type against the annotation afterwards.

Constraint generation rules for patterns have the form
\[\Delta;\ pat \Downarrow_p \langle ty,\ C,\ b\rangle.\]
\noindent and provide the type of the pattern and a set of contraints on that and all
nested patterns. But contrary to constraint generation rules for expressions,
all occurring variables are provided as bound to given parts of the matched
value and accordingly equiped with a type in $b$.

Additionaly there is a fourth kind of constraint generation rules:
\[\Delta;\ pat\ \code{->}\ lexp {\color{Gray}\ |\ rules} \Downarrow_r \langle
ty_p,\ ty_e,\ C,\ u\rangle.\]
\noindent This covers the combination of patterns and expressions as in used in the value
mathing, functional abstraction and exception catching to deal with different
variants of the matched value. The rule arranges the bound variables of the
patterns to be available while generating constraints on the right hand side
expressions and further generates constraints on the equality of the pattern's
as well as the expression's types.

\easyocaml's constraint generation does a second task anon: It checks th
validity of the variables, record constructors, field accesses and type
constructions. The handling of those errors and others are described in the
next section.
