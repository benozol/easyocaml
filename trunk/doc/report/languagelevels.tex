
\section{Language Levels and Teachpacks}
\label{sec:teachpacks}

Language levels are a facility to describe the initial state of the 
EasyOCaml compiler or toplevel system in means of the environment which 
is accessible to the user and the available syntax.  Language levels are 
useful for teaching programming, as they can be designed just for 
specific exercises -- probably providing an easy interface to some 
advanced API and restrictions on the syntactic elements taught so far, 
to avoid syntax errors regarding unknown syntactic elements.

Here is in more detail, what language levels can define: The available 
\emph{syntactic features}.  One can specify the syntactic elements which 
are allowed for patterns, expressions, structure items and type 
declarations, in high detail. Currently, most of these restrictions are 
implemented by deleting the according entries from the grammar (thanks 
to the power of Camlp4!).  However, some features like mandatory type 
annotations for toplevel values are checked afterwards while importing 
the AST to EasyOCaml's AST.

Settings of \emph{path inclusion} and \emph{object loading and opening}.  
Teach packs can specify the directories which are included for searching 
objects, just like the \texttt{-I} command line flag.  A teach pack can 
contain objects itself and the specification which have to be loaded 
(just like putting them on the command line).  Furthermore, teach packs 
can specify which modules are opened on startup.

Teach packs can specify the settings for path inclusion and object 
loading. Whereas only one language level can be loaded, teach packs can 
extend a possible language level.

The user can specify which language level and teach pack to use by the 
\texttt{-easylevel} and \texttt{-easyteachpack} command line parameter 
respectively. EasyOCaml then searches for it in the following 
directories:

There is a global and a user configuration directory. First, EasyOCaml 
searches the user then the global configuration directory.  Here's how 
the global configuration directory is determined (in descending 
preference):

\begin{enumerate}
    \item Environment variable EASYOCAML\_GLOBAL\_DIR
    \item Compile-time option
\end{enumerate}

Here's how the user configuration directory is determined (in descending 
preference):

\begin{enumerate}
    \item Environment variable EASYOCAML\_USER\_DIR
    \item \$HOME/.easyocaml
\end{enumerate}

Layout of the easyocaml configuration directory:

\begin{verbatim}
 language-levels/level-1
                 level-2
                 ...
 teachpacks/tp-1
            tp-2
            ...
\end{verbatim}

Each language level and teach pack contains a module \texttt{LANG\_META} 
which is loaded into EasyOCaml.

The idea of teach packs and language levels is taken from DrScheme.  See 
\href{http://docs.plt-scheme.org/drscheme/extending-drscheme.html}{http://docs.plt-scheme.org/drscheme/extending-drscheme.html}
for more information.



