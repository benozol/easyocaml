
\section{Language Levels and Teachpacks}
\label{sec:teachpacks}

Language levels and teach packs are didactic tools to simplify a language and
its handling for beginners. They were introduced by \drscheme\
\citep{Felleisen98thedrscheme}  with two goals in mind:
First, they split the Scheme language into a ``tower of languages''---that is
different levels of syntactic and semantic richness.
And language levels can specify the available language.
Second, language levels and teach packs can provide arbitrary code to
the user in a simple and encapsulated way. In contrast, teach packs contain only
additional code but can extend a currently active language level.

Language levels and teach pack are a mean to adapt the language to the
knowledge of students at a dedicated state of the class: Reasonable levels of
the tower of languages are e.g.\ a first-order functional language, a
higher-order functional language or a language with imperative features and
mutable date.

To use a full-fledged parser and to check for the syntactical restraints afterwards
would facilitate syntax errors regarding syntactical categories not included in
the current language level---very confusing for beginners. 
\easyocaml, however, directly manipulates the parser to the requirements of the
language level.
The parser itself does not even ``know'' about the syntactical
categories not part of the language level and  does not report errors regarding
to them in the sequel.

\easyocaml\ further grants fine grained control over the available syntax.
Every option for non-terminal nodes in the grammar (structure items, type
declarations  expressions, patterns) can be switched on and off. This can be
even done independently for different usages, e.g. the patterns in value
matching can be configured in a different way than the patterns in functional
abstractions.
This is implemented by deleting the minimal common disabled set of options from
the \camlpf\ grammar.
At a second stage (namely while transforming the \ocaml\ AST into
\easyocaml's), all remaining restrictions are examined.

As mentioned, language levels and teach packs can be used to provide code at
start-up time of the compiler and toplevel loop in a simple manner. They contain
a list of modules which ought to be available to the user. Each goes with an
annotation if they are opened at start-up time.
Section \ref{sec:definelanglevels} shows how to define custom language levels
and teach packs.

The user can specify which language level and teach pack to use by some command
line parameter as described in section~\ref{sec:commandlineflags}.
\easyocaml\ then searches for it in a dedicated directory as described in
section \ref{sec:directory}.

