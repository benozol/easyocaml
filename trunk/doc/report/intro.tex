
\section{Objectives and Introduction}
\label{hd001}
\emph{Objective Caml} \citep{leroy2008} is a programming language which unifies 
functional, imperative and object oriented concepts in a ML-like 
language with a powerful and sound type system.  Its main implementation 
(\href{http://caml.inria.fr}{http://caml.inria.fr}) ships 
with a platform independent byte code compiler and an efficient machine 
code compiler and there are a lot of existing libraries, which make it a 
real multi purpose programming language.

Up to now, OCaml is not a language well suited for \emph{learning and teaching
programming}: It has a very rich type system, but type errors are reported only
by a message and a single location in the code without giving any clues for the
underlying reasons. On the other hand, OCaml comes with tools (e.g.
\citeauthor{stolpmann}'s \emph{findlib}) which make it easy to handle libraries
for developers, but it lacks a fool-proof system to use primed code.

The objectives of this work are, in large, to make OCaml a
programming language better suited for beginners and to teach programming. We
achieve this by

\begin{itemize}
    \item improving OCaml's error messages by providing a modified 
        parser and a new type checker.
    \item equipping OCaml with an infrastructure to make it 
        adaptable for teaching programming, in means of restricting the 
        supported features of the language, or providing code and the 
        startup environment in a simple way of distribution (language 
        levels).
    \item integrating all that into OCaml's original toploop and 
        compiler system to take advantage of existing libraries and 
        OCaml's code generation facilities.
\end{itemize}
The project is hosted at
\href{http://easyocaml.forge.ocamlcore.org}{http://easyocaml.forge.ocamlcore.org}.

\subsection{Supported Language}

\subsection{Similar Projects}
\label{hd001002}
There are some projects which heavily influenced our work:

\citet{Haack_abstracttype} have described and implemented a technique 
to produce more descriptive type error messages in a subset of SML.  
Their work is seminal for constraint based type checking with attention 
on good error reporting and builds the foundation for EasyOCaml's type 
checker.

\emph{Helium} \citep{helium-hw03} is a system for teaching programming 
in Haskell. In a similar manner, type checking is done via constraint 
solving. Furthermore, it features detailed error messages including 
hints how to fix errors based on certain heuristics.

Finally, \emph{DrScheme} \citep{Felleisen98thedrscheme} is a programming 
environment for the Scheme language that is build for teaching 
programming.  It has introduced the concept of language levels and teach 
packs to restrict the syntax and broaden functionality especially for 
exercises. These are included in EasyOCaml, too.

This report has three goals: Firstly, to present EasyOCaml and the 
concepts used and developed for it (target audience: the users) through 
section~\ref{startpart1} --~\ref{endpart1}.
Secondly, formal foundations for the type checker are given the appendix (target
audience: the interested readers) in section~\ref{sec_rules}.
Thirdly, it combines these concepts with the actual implementation and
describes its architecture in code (target audience: the developers) in
section~\ref{impldets}.

\label{hd001001}
EasyOCaml supports a strict subset of OCaml, namely Caml without module 
declarations (called ``Caml-$m$''). Take a look into the file 
easyocaml-features.pdf for more details on the supported features. 
\what

