
\section{Objectives and Introduction}
\label{sec:intro}

\label{hd001}
\emph{Objective Caml} \citep{leroy2008} is a programming language which unifies 
functional, imperative and object oriented concepts in a ML-like 
language with a powerful and sound type system.  Its main implementation 
(\url{http://caml.inria.fr}) ships with a platform independent byte code
compiler and an efficient machine code compiler and there are a lot of existing
libraries, which make it a great multi purpose programming language.

Up to now, OCaml is not a language well suited for \emph{learning} and
\emph{teaching programming}:
It has a very rich type system, but type errors are reported only with few
information on the underlying reasons.
Some practice is necessary to manage these.
On the other hand, OCaml comes with tools (e.g.  \citeauthor{stolpmann}'s
\emph{findlib}) which make it easy to handle libraries for developers, but it
lacks a fool-proof system to use primed code in programming lessons.

The objectives of this work are, in large, to make OCaml a
programming language better suited for beginners and to teach programming. We
achieve this by

\begin{itemize}
    \item improving OCaml's error messages by providing a modified 
        parser and a new type checker.
    \item equipping OCaml with an infrastructure to make it 
        adaptable for teaching programming, in means of restricting the 
        supported features of the language, or providing code and the 
        startup environment in a simple way of distribution (language 
        levels).
    \item integrating all that into OCaml's original toploop and 
        compiler system to take advantage of existing libraries and 
        OCaml's code generation facilities.
\end{itemize}
The project is hosted at
\url{http://easyocaml.forge.ocamlcore.org}  where an online demo of EasyOCaml's
type inference and languge levels can be found, too.

\subsection{Similar Projects}

There are some projects which heavily influenced our work:

\citet{haackwells04} have described and implemented a technique 
to produce more descriptive type error messages in a subset of SML.  
Their work is seminal for constraint based type checking with attention 
on good error reporting and builds the foundation for EasyOCaml's type 
checker.

\emph{Helium} \citep{helium-hw03} is a system for teaching programming 
in Haskell. In a similar manner, type checking is done via constraint 
solving. Furthermore, it features detailed error messages including 
hints how to fix errors based on certain heuristics.

Finally, \emph{DrScheme} \citep{Felleisen98thedrscheme} is a programming 
environment for the Scheme language that is build for teaching 
programming.  It has introduced the concept of language levels and teach 
packs to restrict the syntax and broaden functionality especially for 
exercises.

\easyocaml\ uses and combines ideas of all these project for the attempt to
make \ocaml\ better suited for learning and teaching programming.


\section{Supported Language}
\label{sec:language}

\easyocaml\ targets to be usable not only for the very first steps in
programming, so a reasonable subset of the \ocaml\ language is supported.
This language can be characterized as ``Caml without module declarations'',
hence its designation \camlm.

Unlike simpler functional programming languages like \textit{Scheme} with only a single
syntactic category \emph{expression}, \ocaml\ makes a distinction between 
\emph{structure items}, \emph{expressions} and \emph{patterns}.
Here is the supported language features in more detail, which also can be
pruned by teach packs. See section \ref{sec:grammar} for a complete grammar.


\subsection*{Structure items}

\ocaml\ programs consist of a list of \emph{structure items}, which are used to
declare values, types and exceptions.
\easyocaml\ supports
\begin{itemize}
  \item Optionally parametrized \emph{type declarations} of type synonyms,
    records with optionally mutable fields and variant types.
  \item \emph{exception declarations} like in \ocaml.
  \item optionally recursive (\texttt{rec}) and multiple (\texttt{and})
    \emph{value declarations}, where binding occur with arbitrary patterns.
    Note, that a languge level may require type annotations of those toplevel
    value declarations.
  \item Toplevel evaluations. Note that the toplevel evaluation of \texttt{e} is just
    syntactic sugar for \texttt{let \_ = e}.
\end{itemize}

\subsection*{Core types}
Direct combinations of existing types are called \emph{core types} in \ocaml.
\easyocaml\ allows as core types primitive types (\texttt{int}, \texttt{char},
\texttt{string} and \texttt{float}), free (in type annotations) and bound (in
type declarations) type variables, type arrows (function types), tuples and
type constructors, i.e.\ applications of parametrized types.

\subsection*{Expressions}

Expressions are parts of a program that can be evaluated to a \ocaml\ value and
occur only as part of structure items.
\easyocaml\ supports simple expressions that can be found in \textsl{MiniML},
too, like variables, functions, infix operators, conditionals and variable
binding.

Despite those, it features construction of tuples, records and variants,
conditionals with optional \texttt{else} branch, \texttt{while} and
\texttt{for} loops, sequences of expressions, exception handling (raising and
catching) as well as type annotations.

\subsection*{Patterns}

In \easyocaml, pattern matching is possible in every place where it works in
\ocaml, i.e.\ value matching\ttfootnote{match \ldots\ with \emph{pat} ->
\ldots\ | \ldots}, in variable bindings\ttfootnote{let \emph{pat} = \ldots}, in
functional abstractions with the \texttt{function} keyword\ttfootnote{function
\emph{pat} -> \ldots\ | \ldots} and exception catching\ttfootnote{try \ldots\
with \emph{pat} -> \ldots\ | \ldots}.

Pattern matching works for every possible value in \easyocaml, i.e.\ primitive
values, tuples, variants and records\footnote{version 0.49 lacks the
implementation for pattern matching on the latter} and every can be nested in
every level.



\subsection*{Goals of this report}
This report is split in a less formal part \ref{part:concepts} which is easily readable
and a more formal part which acts as a reference for the user and developer.
This report has four goals:

Firstly to present EasyOCaml and the concepts used in and developed for it
(target audience: the users) through section~2--5.
Then, it describes the usage of the programs \texttt{ecaml} and \texttt{ecamlc}
for usage in section~\ref{sec:manual}.
Afterwards it combines the concepts with the actual implementation and
describes its architecture in code (target audience: the developers) in
section~\ref{impldets}.
Finally, formal foundations for the type checker are given the section~\ref{sec:rules} (target
audience: the interested readers).
