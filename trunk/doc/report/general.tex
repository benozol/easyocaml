
\section{General Functionality of EasyOCaml}
\label{sec:general}

This section covers some general information on \easyocaml's architecture and
its language for orientation in subsequent sections.
It will give a broad overview on \easyocaml's sequence plan.
Every step is elaborated in more detail in subsequent sections, but they might
be easier to understand with the knowledge of their role within \easyocaml.
This section wraps up the a desription of the language supported by \easyocaml.

\subsection{Outline of EasyOCaml's Course}

By large, \easyocaml's additions to the compiler and the toplevel loop do a
fairly similar job: First, both programs parse the command line flags. If they
contain the \texttt{-easy} flag, \easyocaml's type checker is enabled and flags
for defining language levels, teach packs and error printers are accepted and
loaden into \easyocaml.

Then, \easyocaml\ takes over the first stage of the compiler by parsing the
input with a \camlpf\ parser. The parser incorporates modifications by the
language level to the availabel syntax. The resulting abstract syntax tree (AST)
is annotated with fresh type variables for the next stage.

\easyocaml\ then tries to infer the types in the program. It reads the definitions in
the program and generates a set of constraints on the types of every element of
the program by traversing the AST. The constraints on the node's types are based
on the node's usage and their generation is detailed in the section
\ref{sec:typeinfer} and \ref{sec:rules}.
Then, the type checker tries to solve those constraints. If they are unifiable,
the type checker can assign a valid typing on the elements of the AST and passes
control back to \ocaml\footnote{Note, that up to now the program is type checked
again by \ocaml\ and the results are compared to validate our type checker. This
might change in upcoming versions of \easyocaml.}.
Otherwise, \easyocaml\ tries to generate as many conflict sets (type errors) as
possible from the constraints and reports them to the user.

\subsection{Supported Language: Caml$_{-m}$}
\label{sec:language}

\easyocaml\ targets to be usable not only for the very first steps in
programming, so a significant subset of the \ocaml\ language is supported.
This language can be characterized as ``Caml minus module declarations'',
hence its designation: \camlm.

Unlike simpler functional programming languages like \textit{Scheme} with only
a single syntactic category \emph{expression}, \ocaml\ makes a distinction
between \emph{structure items}, \emph{expressions} and \emph{patterns}.
The options for each syntactic category which will be described in the
following, can be pruned by language levels in high detail, as given in
section~\ref{sec:teachpacks}.
See section \ref{sec:grammar} for the complete grammar.


\subsection*{Structure items}

\ocaml\ programs consist of a list of \emph{structure items}, which are used to
declare values, types and exceptions.
\easyocaml\ supports
\begin{itemize}
  \item optionally parametrized \emph{type declarations} of type synonyms,
    records with optionally mutable fields and variant types.
  \item \emph{exception declarations}.
  \item optionally recursive (\texttt{rec}) and multiple (\texttt{and})
    \emph{value declarations}, where bindings occur in arbitrary patterns for
    non-recursive declarations.  Languge levels may enforce mandatory type
    annotations for toplevel value declarations.
  \item \emph{Toplevel evaluations}. Note that a toplevel evaluation of
    \texttt{e} is just syntactic sugar for the dummy value declaration
    \texttt{let \_ = e}.
\end{itemize}

\subsection*{Core types}
Direct combinations of existing types are called \emph{core types} in \ocaml.
\easyocaml\ allows as core types primitive types (\texttt{int}, \texttt{char},
\texttt{string} and \texttt{float}), free (in type annotations) and bound
(within type declarations) type variables, type arrows (function types), tuples
and type constructors, i.e.\ applications of parametrized types.

\subsection*{Expressions}

Expressions are parts of a program which can be evaluated to a \ocaml\ value and
occur only as part of structure items.
\easyocaml\ supports simple expressions that can be found in \textsl{MiniML},
too, like variables, functions, infix operators, conditionals and variable
binding.

Despite those, it features the construction of tuples, records and variants,
conditionals with optional \texttt{else} branch, \texttt{while} and
\texttt{for} loops, sequences of expressions, exception handling (raising and
catching) as well as type annotations.

\subsection*{Patterns}

In \easyocaml, pattern matching is possible in every site where it works in
\ocaml, i.e.\ in value matching\ttfootnote{match \ldots\ with \emph{pat} ->
\ldots\ | \ldots}, in variable bindings\ttfootnote{let \emph{pat} = \ldots}, in
functional abstractions with the \texttt{function} keyword\ttfootnote{function
\emph{pat} -> \ldots\ | \ldots} and exception catching\ttfootnote{try \ldots\
with \emph{pat} -> \ldots\ | \ldots}.

Pattern matching works on every possible value in \easyocaml, i.e.\ primitive
values, tuples, variants and records\footnote{The current version of \easyocaml\
still lacks the implementation for pattern matching on the latter.}, and can be
nested at every
level.

