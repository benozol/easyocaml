
\section{Errors and Error Reporting Adaptibility}
\label{sec:errors}

\easyocaml\ is essentially build for teaching programming.
As such, special attention is paid to the way errors are reported to achieve
the following goals:

Firstly, errors should provide a right amount of details, too few information
is of course insufficient, but also too much information can be confusing. So,
for example in type constructor clashes exactly those locations of the program
should be reported, which are essential to the error.
Delivering more information on the underlying reasons of type errors is exactly
what \easyocaml's type checker is made for.

Secondly, error reporting should be adaptable: Reading errors in a foreign
language can distract or even prevent for a beginner from understanding it.
So internationalization of error messages is necessary.
Furthermore, the error output should be adaptable in its overall structure to
serve as the input for different kinds of presentation, e.g. plain text on
command line, or HTML to display it in a web browser.
Adaption of the language used for error reporting can be achieved by changing 
the environment variable \texttt{LANG}. The formatting can be given by a
command line argument (see section \ref{sec:manual} for details).

The last section has explained the improvements of \easyocaml\ to the type error
messages in the last section and this one will describe the structure  of the
errors handled by \easyocaml\ as well as the changes to the parser to build a
foundation to include more information with the parsing errors below and the
error adaption possibilities subsequently.

\subsection{The Structure of Errors in EasyOCaml}
\label{sec:easyerrors}

While\new parsing and type checking, \easyocaml\ can detect different errors.
Differences are made between the following three classes of errors\footnote{the
file \texttt{ezyErrors.mli} provides a complete and well documented list of
errors}:

The compiler attempts to report errors as late as possible to collect as many
errors as possible. \emph{Common errors} are reported at the very end after
contstraint unification. So type errors (type constructor clashes, clashes of
the arities of tuples and circular types) are of course of this kind.
Furthermore, unbound variables are admittedly detectey during constraint
generation but provided with free type variables, such that constraint
unification is possible but no extra errors are introduced.
Other light errors like the attemt to change the value of immutable record
fields are reported after constraint unification, too, and invalid type
annotations as well.

\emph{Heavy errors} are collected during and reported directly after constraint
generation, because they make it impossible to assign at least a transitionally
type to the term. They include among others:
Incoherent record constructions (fields from different records or several
bindings of a field), several bindings of a variable in a pattern, the usage of
unknown variant constructors or wrong usage of type constructors (unknown or
wrong number of type arguments).

Syntactical errors (parsing errors) and accessing inexistent modules are
\emph{fatal errors} and stop the compiler directly, because it prevents
reasonable further processing of the program.

The next section will describe our changes to \camlpf\ for more detailled
parsing errors.

\subsection{New Errors for Camlp4}

As mentioned, the program code is parsed py a \camlpf\ parser in  \easyocaml.
Unfortunately, \camlpf's error messages are hard coded in English and never
represented in data which prohibits the adaption of format and language while
reporting.  This is because \camlpf\ is a \ocaml\ stream parser in its core,
which requires parsing errors to be reported as exceptions with just a string
for information\ttfootnote[exception ]{Stream.Error of string}.

Nevertheless, we supplied \camlpf\ with a new error reporting system, up 
to now just to make error reporting adaptable, but it should be possible 
now to augment the information of parsing errors with more information on the
state of the parser. A parsing error\ttfootnote[type ]{ParseError.t} is one of
the following:

\begin{description}
    \item[\texttt{Expected~(entry,~opt\_before,~context)}] is raised 
        if when the parser stucks while parsing a phrase: \texttt{entry} 
        describe the categories of the possible, expected subphrases, 
        \texttt{opt\_before} might describe the category of the entry 
        just parsed and \texttt{context} denotes the category of the 
        phrase which contains the entry.
    \item[\texttt{Illegal\_begin~sym}] is raised when the parser is 
        not able to parse the program's toplevel categories given by
        \texttt{sym}.
    \item[\texttt{Failed}] is raised only in 
        \texttt{Camlp4.Struct.Grammar.Fold}.
    \item[\texttt{Specific\_error~err}] Beside the generic parsing 
        errors just mentioned, it is possible to extend the parsing 
        errors per language by ``artificial'' errors which are specific 
        to a language, e.g. currified constructor in OCaml, which is not
        represented in the grammar but checked in code.  (further errors for
        EasyOCaml are specified in subsection \ref{mod:ocamlspecificerrors}.)
\end{description}

How are these errors represented in the string information of the stream error?
Not without a hack which is luckily hidden behind the interface of \camlpf:
Internally, parsing exceptions contain a string of the format
``\texttt{<msg>\textbackslash 000<mrsh>}'' where \texttt{<msg>} is
the usual \camlpf\ error message and \texttt{<mrsh>} is the marshalled parsing
error as just described.
This string is again decomposed in \camlpf's interface function for
parsing\ttfootnote[function ]{Camlp4.Struct.Grammar.Entry.action\_parse}, and reported as
a parsing error\ttfootnote[type ]{ParseError.t} to the user.

And so \camlpf's parsing errors are now represented in data to apply error
reporting adaptabilities and to further extend them by more information on the
parser's state.

\subsection{Adaptibility}

For internationalization of error messages and different structures of error
messages for different display settings, \easyocaml\ provides adaptability of
error messages by a plugin system.  Error reporting plugins should use
\texttt{EzyError}`s internationalized functions to output the error's
description (\texttt{EzyErrors.print*\_desc}) to keep them uniform but can
print it any structure:  Currently, a plain text format is default and a HTML
printer which highlights the locations of an error in source code and a
XML/Sexp printer for usage in an IDE are delivered with EasyOCaml.

The user can register an error printer via the command line flag 
\texttt{-easyerrorprinter}.  The module is dynamically linked and 
registers itself with \texttt{EzyErrors.register} where appropriate 
functions are overwritten.

The following section describes the tools which EasyOCaml provides 
specifically for teaching programming.
