\documentclass
%[handout]
{beamer}
\usepackage{folien}
\usepackage[utf8]{inputenc}
\usepackage{graphicx}
\usepackage{alltt}
\usepackage{myrules}
\usepackage{pgfpages}
%\pgfpagesuselayout{2 on 1}[a4paper, shrink=5mm]


\author{Benus Becker}
\title{EasyOCaml}
\subtitle{}
\date{28. Januar 2009}
\institute{Universität Freiburg
% \\ Institut für Informatik \\ Abtg.\ für Programmiersprachen
}
\keywords{ocaml, types}
\subject{Studienarbeit}

\begin{document}

\frame{\titlepage}

\begin{frame}{Motivation}
  \centerline{\tt let f b x = let y = if b then x in x + y ;;}
  \pause
  \begin{description}
    \item[Objective Caml]\ \\
      \texttt{\# let f b x = let y = if b then x in \underline x + y ;;\\
      Error: This expression has type unit but is here used with type int}
      \pause
    \item[EasyOCaml]\ \\
      \includegraphics<-3>[rotate=90,width=0.75\textwidth]{mightadd}
      \includegraphics<4->[rotate=90,width=0.75\textwidth]{mightadd1}
  \end{description}
\end{frame}

\begin{frame}{Ziele}
  \begin{itemize}
    \item Fehlermeldungen verbessern
      \begin{itemize}
        \item Parser
        \item Typchecker
      \end{itemize}
    \item didaktische Hilfsmittel (Sprachlevels und Teachpacks)
      \begin{itemize}
        \item Einschränkungen der Syntax
        \item Bereitstellung von Code
        \item Anpassbarkeit der Fehlerausgabe
      \end{itemize}
    \item Integration in das existierende OCaml System
      \begin{itemize}
        \item Nutzen der existierenden Codegenerierung
        \item Zugriff auf vorhandene Programmbibliotheken
      \end{itemize}
  \end{itemize}
\end{frame}

\begin{frame}{Verwandte Projekte}
  \begin{description}
    \item[Haack \& Wells (2004)]\ \\
      \begin{itemize}
        \item Constraint-basiertes Typchecken von MiniML
        \item minimale, ausreichende Begründung der Fehler
      \end{itemize}
    \item[Helium]\ \\
      \begin{itemize}
        \item Haskell Implementation ``für Anfänger''
        \item Hinweise zum Lösen der Typfehler
      \end{itemize}
    \item[DrScheme]\ \\
      \begin{itemize}
        \item Standard IDE für Programmierkurse in Scheme
        \item einfacher Debugger
        \item Sprachlevel und Teachpacks
      \end{itemize}
  \end{description}
\end{frame}

\begin{frame}{Unterstützte Sprache}
  Caml$_{-m}$: Caml ohne Moduldeklarationen
  \begin{itemize}
    \item Primitive: \texttt{float}, \texttt{int}, \texttt{bool}, \texttt{string}
    \item Tupel \interitem\ Listen \interitem\ Arrays
    \item Varianten, Records
    \item Deklarationen von \interitem\ (polymorphen) Typen
      \interitem\ Ausnahmen \interitem\ Werten
    \item schachtelbare, auf alle möglichen Werte anwendbare Patterns
    \item Konstruktoren, Ausnahmebehandlung, Konditionale, Abstraktionen, Typannotationen.
  \end{itemize}
  \comment{Einschränkbar durch Sprachlevels}
  \comment{Alle Teile der stdlib ohne format typbar}
\end{frame}

\begin{frame}{Ablauf von EasyOCaml}
  \begin{itemize}
    \item Kommandozeilenparameter: \texttt{-easy}, \texttt{-easyerrorprinter},
      \texttt{-easylevel}, \texttt{-easyteachpack}
    \item Modifikation des Parsers, Laden von Sprachlevels und Teachpacks
    \item Syntaxanalyse mit Camlp4
    \item Constraint-basierte Typinferenz
    \item Zurückführung in den Compiler bzw.\ die REPL
      \comment{bisher typt Ocaml selbst erneut}
  \end{itemize}
\end{frame}

\begin{frame}{Typchecker}
  \framesubtitle{Haack \& Wells, 2004}
  \begin{description}
    \item[OCaml]
      \texttt{\small let f b x = let y = if b then x in \underline x + y}
    \item[EasyOCaml] \texttt{\small function x -> .. if .. then x .. (+)~x ..}
      \comment{fake!}
  \end{description}
   \comment{
     $\mathcal W$ erzeugt und löst Constraints gleichzeitig: Keine Nachprüfung möglich \\
     Trennung von Generierung und Lösung}
  \begin{itemize}
    \item Ziele
      \begin{itemize}
        \item möglichst viele Fehler anzeigen
        \item Fehler sind minimal und vollständig\comment{definieren}
      \end{itemize}
      \pause
    \item Haack \& Wells' Typchecker für MiniML
      \begin{enumerate}
        \item Generierung von Constraints
        \item Lösen von Contraints
        \item Aufzählen der Fehler
        \item Minimierung der Fehler
      \end{enumerate}
      \pause
    \item gleichzeitig mit Constraintgenerierung:\\
      Erkennung unbekannter Variablen
  \end{itemize}
\end{frame}

\begin{frame}{Typchecker}
  \framesubtitle{Erweiterungen für EasyOCaml}
  \begin{itemize}
    \item Syntaktische Kategorien
      \begin{description}
        \item[Deklarationen] $\Delta;\ strit \Downarrow_s \langle \Delta',\ C,\ u\rangle$
        \item[Ausdrücke] $\Delta;\ lexp \Downarrow_e \langle ty,\ C,\ u\rangle$
        \item[Pattern] $\Delta;\ pat \Downarrow_p \langle ty,\ C,\ b\rangle$ 
      \end{description}
    \item Typen: Varianten, Records \\ Beispiel
%\inferrule[Variant] {
%  \Delta\onvar(K) = \langle ty_r,\ [ty_{a,1},\ \dots,\ ty_{a,n}]\rangle \\
%  \etyjudge \Delta {lexp_i} {ty_i} {C_i} {u_i} \text{ for } i=1,\dots,n \\
%  C_0 = \{a \xlongequal l ty_r \} \cup \{ ty_i \xlongequal l ty_{a,i}\ |\ i=1,\dots,n \} \\
%  a \fresh
%} {
%  \etyjudge \Delta {(K\  lexp_1\ \dots\ lexp_n)^l} a {\bigcup_{i=0}^n C_i} {\bigcup_{i=1}^n u_i}
%}
      \[\inferrule[Record Access]
      {
       \etyjudge \Delta {lexp} {ty} C u \\
       \Delta\onrecord(f) = \langle ty_f,\ ty_r,\ \cdot \rangle \\
       C_0 = \{a \xlongequal {l} ty_f,\ ty \xlongequal l ty_r\} \\
       a \text{ fresh}}
       {\etyjudge \Delta {(lexp\code{.}f)^l} a {C_0 \cup C} u}\]
  \end{itemize}
  \comment{mehr mögliche Fehler}
\end{frame}

\begin{frame}{Typannotationen}
  $(lexp : ct)$\\[1em]
  \begin{itemize}
    \item Typinferenz für $lexp$ und Kontext unabhängig
    \item nachträgliche Prüfung der Validität der Annotation
  \end{itemize}
\[\inferrule[Type-Annot]{
\etyjudge \Delta {lexp} {ty} {C_0} u \\
C_1 = \{a \xlongequal l ty',\ ty \succcurlyeq^l ct\} \\
a \text{ fresh} \\
ty' \text{ is a fresh instance of } ct \\
}{
  \etyjudge{\Delta}{(lexp : ct)^l}{a} {C_0 \cup C_1} {u}
}\]
      \comment{Annotation wird als korrekt angenommen}
\end{frame}

%\begin{frame}{Beispiel: Generierung von Typconstraint}
%  \texttt{x + if b then y}
%\end{frame}

\begin{frame}{Fehlerbehandlung}
  Drei Klassen von Fehlern
  \begin{description}
    \item[Einfach] Meldung nach Typrekonstruktion: \\
      Typfehler, unbekannte Variablen
    \item[Schwer] Meldung nach Constraintgenerierung: \\
      ungültige Varianten-, Recordkonstruktion, Variablenbindungen in Patterns
      \comment{Analyse der Validität von Varianten, Records, Typkonstruktionen}
    \item[Fatal] Sofortige Meldung: Syntaxfehler, unbekannte Module
  \end{description}
  Anpassung der Fehlermeldungen
  \begin{itemize}
    \item Lokalisierung der Fehlerbeschreibung (Umgebungsvariable)
    \item Formatierung für unterschiedliche Ausgaben (Plugin)
      \comment{Funktionen werden registriert in dynamisch geladenem Code}
      \singleitem{textbasiert \interitem\ HTML \interitem\ XML}
  \end{itemize}
\end{frame}

\begin{frame}{Sprachlevel und Teachpacks}
  \begin{itemize}
    \item Definition und einfache Auslieferung verfügbarer Sprachkonstrukte
      und vordefiniertem Codes
      \comment{um die Sprache an Übungen und Inhalt von Kursen anzupassen}
    \item Interface
      \begin{itemize}
        \item (De-)Aktivierung von Optionen für die nicht-terminale Deklaration, Ausdruck, Pattern im Parser
        \item Liste von (zu öffnenden) Modulen
      \end{itemize}
%     \begin{description}
%       \item[Sprachspezifikation]\ \\
%         \begin{itemize}
%           \item 
%           \item[$\Rightarrow$]direkte Manipulation des Parsers
%             \comment{keine Meldungen bzgl.\ unbekannter Kategorien}
%           \item für jede Benutzung unabhängig
%         \end{itemize}
%         \comment{
%           Patterns können anders in try-with und match-with sein \\
%           Parser wird auf den ggT eingeschränkt, anderes wird danach getestet
%         }
%       \item[Vordefinierter Code] 
%         \comment{
%           Sprachlevel definieren Sprache und vorgegebenen Code \\
%           Teachpacks enthalten nur Code und können Ll.s erweitern
%         }
%     \end{description}
  \end{itemize}
\end{frame}

\begin{frame}[fragile]{Beispiel Sprachlevel}
  \begin{columns}
    \begin{column}{0.3\linewidth}
  \texttt{lang-fun/mylist.ml}
  \hrule
  \begin{scriptsize}\begin{verbatim}
let (+) = Pervasives.(+)
let (-) = Pervasives.(-)
let ( * ) = Pervasives.( * )
let (/) = Pervasives.(/)

let succ = Pervasives.succ
let pred = Pervasives.pred
  \end{verbatim}\end{scriptsize}
  \texttt{lang-fun/mylist.ml}
  \hrule
  \begin{scriptsize}\begin{verbatim}
let cons h t = h :: t
let iter = List.iter
let map = List.map
let fold = List.fold_left
  \end{verbatim}\end{scriptsize}
    \end{column}
    \begin{column}{0.7\linewidth}
  \texttt{lang-fun/LANG\_META.ml}
  \hrule
  \begin{scriptsize}\begin{verbatim}
open EzyLangLevel
let _ =
  let pr_f = {
    pr_expr_feats = {
      (all_expr_feats true) with
        e_reference_update = false; (* forbid [x := e] *)
        e_record_field_update = false;  (* forbid [x.f <- e] *)
        e_if_then = false;              (* forbid [if e then f] *)
        e_sequence = false;             (* forbid [e;f] *)
    };
    pr_struct_feats = {
      (all_struct_feats true) with
          (* make annotations for global variables mandatory *)
          s_annot_optional = false;   
          s_type = Some {
            (all_type_feats true) with
                (* forbid declaration of mutable record fields *)
                t_record = Some false } } } in
  configure pr_f [
    "Mylist", true;
    "Mypervasives", true;
  ] ["mylist.cmo"; "mypervasives.cmo"]
  \end{verbatim}\end{scriptsize}
    \end{column}
  \end{columns}
\end{frame}

\begin{frame}{Änderungen am Parser}
  EasyOCaml parst mit Camlp4 Parser
  \begin{itemize}
    \item[{\bf\small +}] veränderbar zur Laufzeit
    \item[{\bf\small --}] hart kodierte Fehlermeldung
      \singleitem{Streamparser erlauben nur Zeichenketten als Information}
  \end{itemize}
  Lösung
  \begin{itemize}
    \item Variantentyp beschreibt Fehler: \\
      \singleitem{Ungültiger Anfang \interitem\ Enttäuchte Erwartung \interitem\ sprachspezifischer/künstlicher Fehler}
    \item interne Struktur der Zeichenkette: \texttt{"<msg>\textbackslash 000<mshl>"}
    \item Wiederherstellung des ``gemarshalten'' Fehlers in der Interfacefunktion 
  \end{itemize}
\end{frame}

\begin{frame}{Bisherige und weitere Entwicklung}
  \begin{itemize}
    \item[\checkmark] Typchecker für Teilsprache von OCaml
    \item[\checkmark] Hilfsmittel für die Lehre mit OCaml
    \item[\checkmark] internationalisierte und anpassbare Fehlermeldungen
    \item[\checkmark] erweitertes Fehlersystem für Camlp4
    \item[\checkmark] HTML/XML/sexp Fehlerausgaben
    \item[\checkmark] Integration in original Compiler und REPL
    \item[\checkmark] Portierung auf OCaml 3.11
    \pause
    \item Integration in DrOCaml oder Camelia
    \item Heuristiken/Tipps zum Lösen von Fehlern
    \item Fehlermeldungen mit mehr Informationen versehen
    \item selbstdokumentierende Sprachlevel und Teachpacks
    \item dynamisch getypter Interpreter
  \end{itemize}
\end{frame}

%\begin{frame}{Fazit}
%  \begin{itemize}
%  \end{itemize}
%  \begin{itemize}
%    \item Zweigeteilt: Arbeit am Typchecker -- Integration in OCaml
%    \item moderne Forschungsergebnisse (Haack \& Wells) anwenden
%    \item OCaml verbessern und open source zu arbeiten
%  \end{itemize}
%\end{frame}

\begin{frame}{Quellenangaben}
  \nocite{leroy2008}
  \nocite{haackwells04}
  \nocite{helium-hw03}
  \nocite{Felleisen98thedrscheme}
  \bibliographystyle{chicago}
  \clearpage\bibliography{easyocaml}
\end{frame}

\end{document}
